%%%%%%%%%%%%%%%%%%%%%%%%%%%%%%%%%%%%%%%%%
% Beamer Presentation
% LaTeX Template
% Version 1.0 (10/11/12)
%
% This template has been downloaded from:
% http://www.LaTeXTemplates.com
%
% License:
% CC BY-NC-SA 3.0 (http://creativecommons.org/licenses/by-nc-sa/3.0/)
%
%%%%%%%%%%%%%%%%%%%%%%%%%%%%%%%%%%%%%%%%%

%----------------------------------------------------------------------------------------
%	PACKAGES AND THEMES
%----------------------------------------------------------------------------------------

\documentclass{beamer}

\mode<presentation> {

% The Beamer class comes with a number of default slide themes
% which change the colors and layouts of slides. Below this is a list
% of all the themes, uncomment each in turn to see what they look like.

%\usetheme{default}
%\usetheme{AnnArbor}
%\usetheme{Antibes}
%\usetheme{Bergen}
%\usetheme{Berkeley}
%\usetheme{Berlin}
\usetheme{Boadilla}
%\usetheme{CambridgeUS}
%\usetheme{Copenhagen}
%\usetheme{Darmstadt}
%\usetheme{Dresden}
%\usetheme{Frankfurt}
%\usetheme{Goettingen}
%\usetheme{Hannover}
%\usetheme{Ilmenau}
%\usetheme{JuanLesPins}
%\usetheme{Luebeck}
% \usetheme{Madrid}
%\usetheme{Malmoe}
%\usetheme{Marburg}
%\usetheme{Montpellier}
%\usetheme{PaloAlto}
%\usetheme{Pittsburgh}
%\usetheme{Rochester}
%\usetheme{Singapore}
%\usetheme{Szeged}
%\usetheme{Warsaw}

% As well as themes, the Beamer class has a number of color themes
% for any slide theme. Uncomment each of these in turn to see how it
% changes the colors of your current slide theme.

%\usecolortheme{albatross}
%\usecolortheme{beaver}
%\usecolortheme{beetle}
%\usecolortheme{crane}
%\usecolortheme{dolphin}
%\usecolortheme{dove}
%\usecolortheme{fly}
%\usecolortheme{lily}
%\usecolortheme{orchid}
%\usecolortheme{rose}
%\usecolortheme{seagull}
%\usecolortheme{seahorse}
%\usecolortheme{whale}
%\usecolortheme{wolverine}

%\setbeamertemplate{footline} % To remove the footer line in all slides uncomment this line
%\setbeamertemplate{footline}[page number] % To replace the footer line in all slides with a simple slide count uncomment this line

\setbeamertemplate{navigation symbols}{} % To remove the navigation symbols from the bottom of all slides uncomment this line
}

\usepackage{graphicx} % Allows including images
\graphicspath{ {../img/} }
\usepackage{booktabs} % Allows the use of \toprule, \midrule and \bottomrule in tables
\usepackage{subcaption} % Allows using subfigures
\captionsetup{compatibility=false}
\usepackage{rotating}
\usepackage{pifont}
%
\fboxsep=0pt
\fboxrule=2pt
%


%----------------------------------------------------------------------------------------
%	TITLE PAGE
%----------------------------------------------------------------------------------------

\title[Interface of 3D Reconstruction]{Development and Application of a Description-based Interface for 3D Object Reconstruction} % The short title appears at the bottom of every slide, the full title is only on the title page

\author{Kai Wu} % Your name
\institute[UBC] % Your institution as it will appear on the bottom of every slide, may be shorthand to save space
{
University of British Columbia \\ % Your institution for the title page
\medskip
\textit{kaywu@ece.ubc.ca} % Your email address
}
\date{\today} % Date, can be changed to a custom date

\begin{document}

\begin{frame}
\titlepage % Print the title page as the first slide
\end{frame}

\begin{frame}
\frametitle{Overview} % Table of contents slide, comment this block out to remove it
\tableofcontents % Throughout your presentation, if you choose to use \section{} and \subsection{} commands, these will automatically be printed on this slide as an overview of your presentation
\end{frame}

%------------------------------------------------
\begin{frame}
\frametitle{Overview of Thesis}

\begin{itemize}
\item \textbf{Taxonomy}: change the way of viewing algorithms, not from an algorithmic viewpoint, but from a problem-oriented (object-centered) viewpoint. More specifically, each algorithm is mapped to a volume of a $n-$dimensional problem space.
\item \textbf{Description}: how to describe a sub-space in the $n-$dimensional problem space so that this sub-space(conditions) that an algorithm maps to is well defined;
\item \textbf{Mapping}: find the mapping between each algorithm and this sub-space in the $n-$dimensional problem space.
\item \textbf{Interpretation}: test the mapping using synthetic and real-world objects.
\end{itemize}
\end{frame}

%----------------------------------------------------------------------------------------
%	PRESENTATION SLIDES
%----------------------------------------------------------------------------------------
%------------------------------------------------
\section{Contribution} % Sections can be created in order to organize your presentation into discrete blocks, all sections and subsections are automatically printed in the table of contents as an overview of the talk
%------------------------------------------------

\begin{frame}
\frametitle{Overview} % Table of contents slide, comment this block out to remove it
\tableofcontents[currentsection,currentsubsection, 
    hideothersubsections, 
    sectionstyle=show/shaded,] % Throughout your presentation, if you choose to use \section{} and \subsection{} commands, these will automatically be printed on this slide as an overview of your presentation
\end{frame}

%------------------------------------------------
\begin{frame}
\frametitle{Contribution}
Development of an interface for 3D reconstruction problem, which hides algorithmic details and allows users to describe conditions surrounding the problem. This description can be interpreteded so that an appropriate algorithm is chosen to reconstruction a successful result.
\end{frame}

%------------------------------------------------
\begin{frame}
\frametitle{Contribution}
This contribution is significant because:
\begin{itemize}
\item No single algorithm can work for a diverse categories of objects. The interface, to some extent, can cover a wider range of object categories by incorporating multiple algorithms.
\item An description is provides that hides the algorithmic details, thus understanding of the algorithm, or conditions to apply a specific algorithm is not a prerequisite.
\end{itemize}
\end{frame}

%------------------------------------------------
\section{Related Work} % Sections can be created in order to organize your presentation into discrete blocks, all sections and subsections are automatically printed in the table of contents as an overview of the talk
%------------------------------------------------

\begin{frame}
\frametitle{Overview} % Table of contents slide, comment this block out to remove it
\tableofcontents[currentsection,currentsubsection, 
    hideothersubsections, 
    sectionstyle=show/shaded,] % Throughout your presentation, if you choose to use \section{} and \subsection{} commands, these will automatically be printed on this slide as an overview of your presentation
\end{frame}

%------------------------------------------------
\section{A Taxonomy to 3D Reconstruction} % Sections can be created in order to organize your presentation into discrete blocks, all sections and subsections are automatically printed in the table of contents as an overview of the talk
%------------------------------------------------

\begin{frame}
\frametitle{Overview} % Table of contents slide, comment this block out to remove it
\tableofcontents[currentsection,currentsubsection, 
    hideothersubsections, 
    sectionstyle=show/shaded,] % Throughout your presentation, if you choose to use \section{} and \subsection{} commands, these will automatically be printed on this slide as an overview of your presentation
\end{frame}

%------------------------------------------------
\begin{frame}
\frametitle{Taxonomy}
\begin{itemize}
\item \textit{algorithm-centered} taxonomy categorizes algorithms based on algorithmic details, as discussed in \textbf{Related Work};
\item \textit{object-centered} taxonomy categorizes algorithms based on the problem conditions that the algorithm can reliably works under.
\end{itemize}

\begin{figure}[h]
\includegraphics[width=0.7\textwidth]{taxo/obj_class}
\end{figure}

\end{frame}

%------------------------------------------------
\begin{frame}
\frametitle{Taxonomy (cont'd)}
Analyze the conditions of each category of algorithms based on literature reports
\begin{table}
\begin{tabular}{l|ccccc}
\toprule
Algo. & Texture & Brightness & Reflectance & Roughness & Concavity\\
\midrule
MVS & Textured & - & Mixed & - & -\\
SfS & - & Bright & Lambertian & - & Convex\\
L PS &  - & Bright & Lambertian & - & Convex\\
NL PS & - & Bright & Mixed & - & Convex\\
SL & - & Bright & Diffuse & - & Convex\\
VH & - & - & - & - & -\\
\bottomrule
\end{tabular}
\caption{Problem conditions of six categories of algorithms based on literature reports. ``L'' stands for Lambertian, and ``NL'' for Non-Lambertian.}
\end{table}

\end{frame}

%------------------------------------------------
\begin{frame}
\frametitle{Taxonomy (cont'd)}
The \textit{object-centered} taxonomy of the six categories of algorithms.
\begin{figure}[!htbp]
\includegraphics[width=0.6\textwidth]{taxo/six_class}
\end{figure}
\end{frame}

% %------------------------------------------------
% \begin{frame}
% \frametitle{Taxonomy (cont'd)}

% Select algorithms:
% \begin{table}[!ht]
%   \centering
%   \begin{tabular}{p{2cm}||p{9cm}}
%   \hline
%   \textbf{Algo. class} & \textbf{Technique}\\
%   \hline
%   SfS & Horn~\cite{horn1970shape}\\
%   MVS & Furukawa~\cite{furukawa2010accurate}, Goesele~\cite{goesele2006multi}, Vogiatzis~\cite{vogiatzis2007multiview}, \\
%       & Hern{\'a}ndez~\cite{esteban2004silhouette}, Faugeras~\cite{faugeras2002variational}\\
%   Lamberian PS & Woodham~\cite{woodham1980photometric}, Hayakawa~\cite{hayakawa1994photometric}, Belhumeur~\cite{belhumeur1999bas}, \\
%       & Alldrin~\cite{alldrin2007resolving}\\
%   Non Lambertian PS & Coleman~\cite{coleman1982obtaining}, Barsky~\cite{barsky20034}, Schluns~\cite{schluns1993photometric}, Sato~\cite{sato1994temporal}, \\
%       & Mallick~\cite{mallick2005beyond}, Alldrain~\cite{alldrin2008photometric}, Goldman~\cite{goldman2010shape}, Silver~\cite{silver1980determining}, \\
%       & Hertzmann~\cite{hertzmann2005example}, Zickler~\cite{zickler2002helmholtz}\\
%   % MVPS & \checkmark & \\
%   SL & Inokuchi~\cite{inokuchi1984range}\\
%   VH & Szeliski~\cite{szeliski1993rapid}, Matusik~\cite{matusik2002efficient}, Tarini~\cite{tarini2002marching}\\
%   \hline
%   \end{tabular}
% \end{table}

% \end{frame}

%------------------------------------------------
\section{A Description to 3D Reconstruction} % Sections can be created in order to organize your presentation into discrete blocks, all sections and subsections are automatically printed in the table of contents as an overview of the talk
%------------------------------------------------

\begin{frame}
\frametitle{Overview} % Table of contents slide, comment this block out to remove it
\tableofcontents[currentsection,currentsubsection, 
    hideothersubsections, 
    sectionstyle=show/shaded,] % Throughout your presentation, if you choose to use \section{} and \subsection{} commands, these will automatically be printed on this slide as an overview of your presentation
\end{frame}

% %------------------------------------------------
% \begin{frame}
% \frametitle{Description}

% We've discussed the characteristics/dimensions of the problem space, then the visual and geometric properties that would affect those dimensions are discussed. These properties forms the basis of the `language' of the description. This chapter consists of four parts: definition, model, representation, expression.

% \begin{itemize}
% \item Definition: segment (cue), scell (property), photo-consistency;
% \item Model: use the dimensions of the problem space as the bases of the model;
% \item Representation:
% 	\begin{itemize}
% 	\item segment: pixel, window area, edge, silhouette
% 	\item scell: voxel, 3d point/patch, curve, contour edge
% 	\item cue: stereo correspondence, intensity (variation)/shadow, silhouette (curvature)
% 	\item property: texture, albedo, specular, roughness, orientation/curvature
% 	\item influence of properties on cues: albedo, specular, roughness$\Rightarrow$ intensity
% 	\end{itemize}
% \item Expression:
% 	\begin{itemize}
% 	\item one example for each class
% 	\end{itemize}
% \end{itemize}

% \end{frame}

%------------------------------------------------
\begin{frame}
\frametitle{Description: model and representations}
\begin{table}[!htbp]
  \centering
  \begin{tabular}{l|l}
  \toprule
  \textbf{Model} & \textbf{Representation}\\
  \midrule
  Texture & \textit{Texture randomness}\\
  Lightness & \textit{Albedo}\\
  Specularity & \textit{Specular/diffuse ratio}\\
  Roughness & \textit{SD of facet slopes}\\
  Concavity & \textit{Surface curvature}\\
  \bottomrule
  \end{tabular}
  \caption{A Model and corresponding representations of the 3D reconstruction problem.}
\end{table}

\end{frame}

%------------------------------------------------
\begin{frame}
\frametitle{Description: expression}
We use three discrete scales to parameterize these properties: \textit{low} (0.2), \textit{medium} (0.5), and \textit{high} (0.8).
\begin{table}[!htbp]
  \centering
  \begin{tabular}{l*{4}{p{1cm}}l}
  \toprule
  \textbf{Object} & Texture & Albedo & Specular & Rough & \textbf{Label}\\
  \midrule
  Class 1 & low/med & high & low/med & high & Tl-B-D-R\\
  Class 2 & low/med & high & high & low/med & Tl-B-M-S\\
  Class 3 & high & high & low/med & high & T-B-D-R\\
  Class 4 & high & high & high & low/med & T-B-M-S\\
  Class 5 & high & low/med & low/med & high & T-D-D-R\\
  Class 6 & high & low/med & high & low/med & T-D-M-S\\
  \bottomrule
  \end{tabular}
  \caption{Expression of the reconstruction problem for the object classss.}
  \label{tab:express}
\end{table}

\end{frame}

%------------------------------------------------
\section{A Mapping to 3D Reconstruction} % Sections can be created in order to organize your presentation into discrete blocks, all sections and subsections are automatically printed in the table of contents as an overview of the talk
%------------------------------------------------

\begin{frame}
\frametitle{Overview} % Table of contents slide, comment this block out to remove it
\tableofcontents[currentsection,currentsubsection, 
    hideothersubsections, 
    sectionstyle=show/shaded,] % Throughout your presentation, if you choose to use \section{} and \subsection{} commands, these will automatically be printed on this slide as an overview of your presentation
\end{frame}

%------------------------------------------------
\begin{frame}
\frametitle{Mapping}

Investigate the problem conditions under which the algorithms can reliably work. This strucure of this chapter is as follows

\begin{itemize}
\item Establsh the \textit{effective problem domain} (EPD): cope with large variation in material and shape.
\item Evaluate within EPD: evaluate algorithmic performance within EPD
\item Derive mapping
\end{itemize}

\end{frame}

%------------------------------------------------
\begin{frame}
\frametitle{Mapping: setup}

Use Blender's physical-based render engine, Cycles to generate synthetic datasets.
% \begin{figure}[!htbp]
% \centering
% \begin{tabular}{ccc}
% \includegraphics[width = 0.3\textwidth]{mapping/setup/mvs_setup.jpg}&
% \includegraphics[width = 0.3\textwidth]{mapping/setup/ps_setup.jpg}&
% \includegraphics[width = 0.3\textwidth]{mapping/setup/sl_setup.jpg}
% (a). MVS & (b). PS & (c). SL\\
% \caption{Synthetic setups.}
% \end{tabular}
% \end{figure}

\end{frame}

%------------------------------------------------
\begin{frame}
\frametitle{Mapping: algorithms and baseline}

\begin{table}[!htbp]
\centering
\begin{tabular}{p{4cm}|c|c|c|c}
\toprule
Technique & Texture & Albedeo & Specular & Roughness\\
\midrule
PMVS: patch-based, seed points propagation MVS. & High & - & Low & -\\
\midrule
EPS: example-based Photometric Stereo & - & High & Low & High \\
\midrule
GSL: Gray-code Structured Light technique & - & High & Low & High\\
\midrule\midrule
VH: volumetric Visual Hull. & - & - & - & -\\
\midrule
LLS-PS: linear least squares Photometric Stereo. & - & High & Low & High\\
\bottomrule
\end{tabular}
\caption{Summary of the selected and baseline algorithms.}
\end{table}

\end{frame}

%------------------------------------------------
\begin{frame}
\frametitle{Mapping: quantitative measures and criteria}

\begin{itemize}
\item accuracy: the distance $d$ such that $X\%$ of the points on $R$ are within distance $d$ of $G$ is considered as accuracy;
\item completeness: the percentage of $G$ that is reconstructed by $R$;
\item angular error: angle between the estimated and ground truth normal, \textit{i.e.,} $arccos(n_g^T n)$.
\end{itemize}

\begin{figure}[!htbp]
\begin{tabular}{cc}
\includegraphics[width=0.6\textwidth]{mapping/qm_acc_cmp} &
\includegraphics[width=0.17\textwidth]{mapping/qm_ang_error}
\end{tabular}
\end{figure}

\end{frame}

%------------------------------------------------
\begin{frame}
\frametitle{Mapping: EPD of PMVS}

\begin{table}[!htbp]
  \centering
  \caption{Problem conditions for establishing the \textit{effective problem domain} of PMVS.}
  \begin{tabular}{l*{4}{c}}
  \hline
  \textbf{Group} & Texture & Albedo & Specular & Roughness\\
  \hline
  \textbf{(a)} & [0.2, 0.8] & [0.2, 0.8] & 0.0 & 0.0\\
  \textbf{(b)} & [0.2, 0.8] & 0.8 & [0.2, 0.8] & 0.0\\
  \textbf{(c)} & [0.2, 0.8] & 0.8 & 0.0 & [0.2, 0.8]\\
  \textbf{(d)} & 0.8 & [0.2, 0.8] & [0.2, 0.8] & 0.0\\
  \textbf{(e)} & 0.8 & [0.2, 0.8] & 0.0 & [0.2, 0.8]\\
  \textbf{(f)} & 0.8 & 0.8 & [0.2, 0.8] & [0.2, 0.8]\\
  \hline
  \end{tabular}
\end{table}

\begin{figure}[!htbp]
\begin{tabular}{ccc}
\includegraphics[width=0.25\textwidth]{mapping/depend_check/mvs_tex_alb}&
\includegraphics[width=0.25\textwidth]{mapping/depend_check/mvs_tex_spec}&
\includegraphics[width=0.25\textwidth]{mapping/depend_check/mvs_alb_spec}\\
(a) & (b) & (d)\\
\end{tabular}
\end{figure}

\end{frame}

%------------------------------------------------
\begin{frame}
\frametitle{Mapping: EPD of PMVS (cont'd)}

\textbf{(a). Texture and Albedo}

\textbf{(b). Texture and Specularity}
\begin{figure}[!htbp]
\begin{tabular}{ccc}
\includegraphics[width=0.25\textwidth]{mapping/mvs_spec/mvs_spec}&
\includegraphics[width=0.25\textwidth]{mapping/mvs_spec/mvs_spec_01}&
\includegraphics[width=0.25\textwidth]{mapping/mvs_spec/mvs_spec_00}\\
(a) Image formation & (b) $V_1$ & (c) $V_2$\\
\end{tabular}
\caption{(a) shows the reflection of light off a specular surface. $V_1$ received the diffuse component while $V_2$ receives the specular component. (b), (c) shows the images observed from these two views. The specular area (red circle) observed in $V_2$ is visible in $V_1$.}
\end{figure}

\end{frame}

%------------------------------------------------
\begin{frame}
\frametitle{Mapping: EPD of PMVS (cont'd)}

\textbf{(d). Albedo and Specularity}
\begin{figure}[!htbp]
\centering
\begin{tabular}{ccc}
\includegraphics[width=0.25\textwidth]{mapping/mvs_alb_spec/alb_spec_0202}&
\includegraphics[width=0.25\textwidth]{mapping/mvs_alb_spec/alb_spec_0205}&
\includegraphics[width=0.25\textwidth]{mapping/mvs_alb_spec/alb_spec_0208}\\
(a) spec: 0.2 & (b) spec: 0.5 & (c) spec: 0.8\\
\includegraphics[width=0.25\textwidth]{mapping/mvs_alb_spec/alb_spec_0202}&
\includegraphics[width=0.25\textwidth]{mapping/mvs_alb_spec/alb_spec_0502}&
\includegraphics[width=0.25\textwidth]{mapping/mvs_alb_spec/alb_spec_0802}\\
(d) alb: 0.2 & (e) alb: 0.5 & (f) alb: 0.8\\
\end{tabular}
\caption{(a)-(c). The albedo is set as 0.2, (d)-(f). The specularity is set as 0.2. According to energy conservation, as the specular component increases, the diffuse component decreases.}
\end{figure}

\end{frame}

%------------------------------------------------
\begin{frame}
\frametitle{Mapping: \textit{effective properties} of PMVS}

\begin{table}[!htbp]
  \centering
  \begin{tabular}{l*{4}{c}}
  \hline
  \textbf{Metric} & Texture & Albedo & Specular & Roughness\\
  \hline
  Accuracy & \checkmark & \checkmark & \checkmark & \ding{55}\\
  Completeness & \checkmark & \checkmark & \checkmark & \ding{55}\\
  \hline
  \end{tabular}
  \caption{The \textit{effective problem domain} of PMVS in terms of accuracy and completeness.}
\end{table}

\end{frame}

%------------------------------------------------
\begin{frame}
\frametitle{Mapping: mapping construction}

\begin{itemize}
\item accuracy and completeness improves consistently as the \textit{texture} level increases. 
\item Accuracy and completeness results deteriorate consistently as \textit{specularity} increases, and this negative impact is most significant when texture level is medium or albedo value is low. 
\item The effect of \textit{albedo} on a surface with low texture is negligible. However, albedo has a noticeably more significant positive impact on completeness as the texture of a specular surface increases.
\end{itemize}

\begin{figure}[!htbp]
\begin{tabular}{ccc}
\includegraphics[width=0.25\textwidth]{mapping/training/mvs_train_spec_02}&
\includegraphics[width=0.25\textwidth]{mapping/training/mvs_train_spec_05}&
\includegraphics[width=0.25\textwidth]{mapping/training/mvs_train_spec_08}\\
(a) & (b) & (c)\\
% \includegraphics[width=0.25\textwidth]{mapping/training/mvs_train_tex_02}&
% \includegraphics[width=0.25\textwidth]{mapping/training/mvs_train_tex_05}&
% \includegraphics[width=0.25\textwidth]{mapping/training/mvs_train_tex_08}\\
% (d) & (e) & (f)\\
% \includegraphics[width=0.25\textwidth]{mapping/training/mvs_train_alb_02}&
% \includegraphics[width=0.25\textwidth]{mapping/training/mvs_train_alb_05}&
% \includegraphics[width=0.25\textwidth]{mapping/training/mvs_train_alb_08}\\
% (g) & (h) & (i)\\
\end{tabular}
\caption{Performance of PMVS under varied conditions of changing property values. The baseline method serves as the guidelines to determine the performance of PMVS.}
\label{fig:mvs_training}
\end{figure}

\end{frame}

%------------------------------------------------
\begin{frame}
\frametitle{Mapping: mapping construction (cont'd)}

\begin{table}[!htbp]
  \centering
  \begin{tabular}{l*{4}{c}}
  \toprule
  \textbf{Metric} & Texture & Albedo & Specular & Roughness\\
  \midrule
  Accuracy & 0.5 & 0.5 & 0.2 & -\\
  \&Completeness & 0.5 & 0.8 & 0.2 & -\\
           & 0.5 & 0.8 & 0.5 & -\\
           & 0.8 & 0.2 & 0.2 & -\\
           & 0.8 & 0.5 & 0.2 & -\\
           & 0.8 & 0.8 & 0.2 & -\\
           & 0.8 & 0.5 & 0.5 & -\\
           & 0.8 & 0.8 & 0.5 & -\\
           & 0.8 & 0.5 & 0.8 & -\\
           & 0.8 & 0.8 & 0.8 & -\\
  \bottomrule
  \end{tabular}
  \caption{The working problem conditions of PMVS in terms of the two metrics \textit{accuracy} and \textit{completeness}.}
  \label{tab:mvs_training_result}
\end{table}

\end{frame}

%------------------------------------------------
\section{An Interpretation of 3D Reconstruction} % Sections can be created in order to organize your presentation into discrete blocks, all sections and subsections are automatically printed in the table of contents as an overview of the talk
%------------------------------------------------

\begin{frame}
\frametitle{Overview} % Table of contents slide, comment this block out to remove it
\tableofcontents[currentsection,currentsubsection, 
    hideothersubsections, 
    sectionstyle=show/shaded,] % Throughout your presentation, if you choose to use \section{} and \subsection{} commands, these will automatically be printed on this slide as an overview of your presentation
\end{frame}

%------------------------------------------------
\begin{frame}
\frametitle{Interpretation}

Key Evaluation Questions:
\begin{itemize}
\item Is the mapping robust to changes of the shape of objects?
\item Can the proof-of-concept interpreter return a satisfactory reconstruction given the correct description?
\end{itemize}

\end{frame}

%------------------------------------------------
\begin{frame}
\frametitle{Interpretation: evaluation of mapping}

\begin{figure}[!htbp]
\centering
\begin{tabular}{cc}
  \includegraphics[width=0.25\textwidth]{interp/synth_data/bottle/bottle_mvs}&
  \includegraphics[width=0.25\textwidth]{interp/synth_data/bottle/bottle_ps}\\
  MVS & PS\\
  \includegraphics[width=0.25\textwidth]{interp/synth_data/bottle/bottle_sl}&
  \includegraphics[width=0.25\textwidth]{interp/synth_data/bottle/bottle_ps_gt}\\
  SL & Normal groundtruth\\
\end{tabular}
\end{figure}

\end{frame}

%------------------------------------------------
\begin{frame}
\frametitle{Interpretation: evaluation of mapping}

\begin{figure}[!htbp]
\centering
\begin{tabular}{p{2cm}|*{4}{p{2cm}}}
  Mapping & Quantitative results & ~ & Qualitative results & ~\\
  \hline
  EPS, GSL & 
  \includegraphics[width=0.15\textwidth]{interp/synth_data/bottle/bottle_02080208}&
  \includegraphics[width=0.15\textwidth]{interp/synth_data/bottle/bottle_mvs_02080208.png}&
  \fcolorbox{green}{white}{\includegraphics[width=0.15\textwidth]{interp/synth_data/bottle/bottle_ps_02080208.png}}&
  \fcolorbox{green}{white}{\includegraphics[width=0.15\textwidth]{interp/synth_data/bottle/bottle_sl_02080208.png}}\\
  & \multicolumn{4}{c}{(a). tex(0.2), alb(0.8), spec(0.2), rough(0.8)}\\
  EPS &
  \includegraphics[width=0.15\textwidth]{interp/synth_data/bottle/bottle_02080502}&
  \includegraphics[width=0.15\textwidth]{interp/synth_data/bottle/bottle_mvs_02080502.png}&
  \fcolorbox{green}{white}{\includegraphics[width=0.15\textwidth]{interp/synth_data/bottle/bottle_ps_02080502.png}}&
  \includegraphics[width=0.15\textwidth]{interp/synth_data/bottle/bottle_sl_02080502.png}\\
  & \multicolumn{4}{c}{(b). tex(0.2), alb(0.8), spec(0.5), rough(0.2)}\\
  \hline
  ~ & ~ & MVS & PS & SL\\
\end{tabular}
\end{figure}

\end{frame}

%------------------------------------------------
\begin{frame}
\frametitle{Interpretation: evaluation of mapping (cont'd)}

\begin{figure}[!htbp]
\centering
\begin{tabular}{p{2cm}|*{4}{p{2cm}}}
  Mapping & Quantitative results & ~ & Qualitative results & ~\\
  \hline
  PMVS, EPS, GSL&
  \includegraphics[width=0.15\textwidth]{interp/synth_data/bottle/bottle_08080208}&
  \fcolorbox{green}{white}{\includegraphics[width=0.15\textwidth]{interp/synth_data/bottle/bottle_mvs_08080208.png}}&
  \fcolorbox{green}{white}{\includegraphics[width=0.15\textwidth]{interp/synth_data/bottle/bottle_ps_08080208.png}}&
  \fcolorbox{green}{white}{\includegraphics[width=0.15\textwidth]{interp/synth_data/bottle/bottle_sl_08080208.png}}\\
  & \multicolumn{4}{c}{(c). tex(0.8), alb(0.8), spec(0.2), rough(0.8)}\\
  PMVS, EPS&
  \includegraphics[width=0.15\textwidth]{interp/synth_data/bottle/bottle_08080502}&
  \fcolorbox{green}{white}{\includegraphics[width=0.15\textwidth]{interp/synth_data/bottle/bottle_mvs_08080502.png}}&
  \fcolorbox{green}{white}{\includegraphics[width=0.15\textwidth]{interp/synth_data/bottle/bottle_ps_08080502.png}}&
  \includegraphics[width=0.15\textwidth]{interp/synth_data/bottle/bottle_sl_08080502.png}\\
  & \multicolumn{4}{c}{(d). tex(0.8), alb(0.8), spec(0.5), rough(0.2)}\\
  \hline
  ~ & ~ & MVS & PS & SL\\
\end{tabular}
\end{figure}

\end{frame}

%------------------------------------------------
\begin{frame}
\frametitle{Interpretation: evaluation of interpreter}

\begin{figure}[!htbp]
\centering
\begin{tabular}{lccccr}
\toprule
Desc \# & Barrel & Vase0 & Bust & Vase1 & Selected Algo.\\
\midrule
1 & 
\fcolorbox{green}{white}{\raisebox{-.5\height}{\includegraphics[width=0.1\textwidth]{interp/synth_interp/barrel_sl}}}&
\raisebox{-.5\height}{\includegraphics[width=0.1\textwidth]{interp/synth_interp/vase2_sl}}&
\raisebox{-.5\height}{\includegraphics[width=0.1\textwidth]{interp/synth_interp/beethoven_sl}}&
\raisebox{-.5\height}{\includegraphics[width=0.1\textwidth]{interp/synth_interp/vase0_sl}}&
GSL\\
2 &
\raisebox{-.5\height}{\includegraphics[width=0.1\textwidth]{interp/synth_interp/barrel_mvs}}&
\fcolorbox{green}{white}{\raisebox{-.5\height}{\includegraphics[width=0.1\textwidth]{interp/synth_interp/vase2_mvs}}}&
\raisebox{-.5\height}{\includegraphics[width=0.1\textwidth]{interp/synth_interp/beethoven_mvs}}&
\raisebox{-.5\height}{\includegraphics[width=0.1\textwidth]{interp/synth_interp/vase0_mvs}}&
PMVS\\
3 & 
\raisebox{-.5\height}{\includegraphics[width=0.1\textwidth]{interp/synth_interp/barrel_sl}}&
\raisebox{-.5\height}{\includegraphics[width=0.1\textwidth]{interp/synth_interp/vase2_sl}}&
\fcolorbox{green}{white}{\raisebox{-.5\height}{\includegraphics[width=0.1\textwidth]{interp/synth_interp/beethoven_sl}}}&
\raisebox{-.5\height}{\includegraphics[width=0.1\textwidth]{interp/synth_interp/vase0_sl}}&
GSL\\
4 & 
\raisebox{-.5\height}{\includegraphics[width=0.1\textwidth]{interp/synth_interp/barrel_ps}}&
\raisebox{-.5\height}{\includegraphics[width=0.1\textwidth]{interp/synth_interp/vase2_ps}}&
\raisebox{-.5\height}{\includegraphics[width=0.1\textwidth]{interp/synth_interp/beethoven_ps}}&
\fcolorbox{green}{white}{\raisebox{-.5\height}{\includegraphics[width=0.1\textwidth]{interp/synth_interp/vase0_ps}}}&
EPS\\
\bottomrule
\end{tabular}
\caption{The evaluation of interpreter using synthetic objects. The first column presents the description provided to the interpreter. Description $i$ matches with condition $i$ in Table~\ref{tab:synth_prop_list}. The last column is the algorithm selected by the interpreter. The object of which condition matches the description is labeled in green rectangle. Since the interpreter would return a successful reconstruction given a description that matches the condition, the quality of reconstruction of the labeled objects indicates success/failure of the interpreter.}
\label{fig:synth_results}
\end{figure}

\end{frame}

%------------------------------------------------
\begin{frame}
\frametitle{Interpretation: real-world objects}

\begin{figure}[!htbp]
\centering
\begin{tabular}{c|*{4}{p{2cm}}}
\hline
class \# & 1 & 2 & 3\&4 & 5\&6\\
\hline
  & textureless & textureless & textured & textured\\
description & diffuse & mixed d/s & diffuse & mixed d/s\\
  & bright & bright & dark/bright & dark/bright\\
\hline
object & 
\raisebox{-.5\height}{\includegraphics[width=0.15\textwidth]{interp/real_world_img/statue/statue}} &
\raisebox{-.5\height}{\includegraphics[width=0.15\textwidth]{interp/real_world_img/cup/cup}} &
\raisebox{-.5\height}{\includegraphics[width=0.15\textwidth]{interp/real_world_img/pot/pot}} &
\raisebox{-.5\height}{\includegraphics[width=0.15\textwidth]{interp/real_world_img/vase/vase}}\\
\hline
label & (a) & (b) & (c) & (d)\\
\end{tabular}
\caption{The rerepsentatives of the six classes of objects used for evaluation.}
\label{fig:test_real_world_6class}
\end{figure}

\end{frame}

%------------------------------------------------
\begin{frame}
\frametitle{Interpretation: evaluation of interpreter (cont'd)}

\begin{figure}[!htbp]
\centering
\begin{tabular}{lccccr}
\toprule
Desc \# & Pot & Vase & Statue & Cup & Selected Algo.\\
\midrule
1 &
\fcolorbox{green}{white}{\raisebox{-.5\height}{\includegraphics[width=0.1\textwidth]{interp/real_interp/pot/pot_sl}}}&
\raisebox{-.5\height}{\includegraphics[width=0.1\textwidth]{interp/real_interp/vase/vase_sl}}&
\raisebox{-.5\height}{\includegraphics[width=0.1\textwidth]{interp/real_interp/statue/statue_sl}}&
\raisebox{-.5\height}{\includegraphics[width=0.1\textwidth]{interp/real_interp/cup/cup_sl}}&
GSL\\
2 &
\raisebox{-.5\height}{\includegraphics[width=0.1\textwidth]{interp/real_interp/pot/pot_mvs}}&
\fcolorbox{green}{white}{\raisebox{-.5\height}{\includegraphics[width=0.1\textwidth]{interp/real_interp/vase/vase_mvs}}}&
\raisebox{-.5\height}{\includegraphics[width=0.1\textwidth]{interp/real_interp/statue/statue_mvs}}&
\raisebox{-.5\height}{\includegraphics[width=0.1\textwidth]{interp/real_interp/cup/cup_mvs}}&
PMVS\\
3 &
\raisebox{-.5\height}{\includegraphics[width=0.1\textwidth]{interp/real_interp/pot/pot_sl}}&
\raisebox{-.5\height}{\includegraphics[width=0.1\textwidth]{interp/real_interp/vase/vase_sl}}&
\fcolorbox{green}{white}{\raisebox{-.5\height}{\includegraphics[width=0.1\textwidth]{interp/real_interp/statue/statue_sl}}}&
\raisebox{-.5\height}{\includegraphics[width=0.1\textwidth]{interp/real_interp/cup/cup_sl}}&
GSL\\
4 &
\raisebox{-.5\height}{\includegraphics[width=0.1\textwidth]{interp/real_interp/pot/pot_ps}}&
\raisebox{-.5\height}{\includegraphics[width=0.1\textwidth]{interp/real_interp/vase/vase_ps}}&
\raisebox{-.5\height}{\includegraphics[width=0.1\textwidth]{interp/real_interp/statue/statue_ps}}&
\fcolorbox{green}{white}{\raisebox{-.5\height}{\includegraphics[width=0.1\textwidth]{interp/real_interp/cup/cup_ps}}}&
EPS\\
\bottomrule
\end{tabular}
\caption{The evaluation of interpreter using real-world objects. The first column presents the description provided to the interpreter. Description $i$ matches with condition $i$ in Table~\ref{tab:real_data_prop_list}. The last column is the algorithm selected by the interpreter. The object of which the condition matches the description is labeled in green rectangle. Since the interpreter would return a successful reconstruction given a description that matches the condition, the quality of reconstruction of the labeled objects indicate the success/failure of the interpreter.}
\label{fig:real_results}
\end{figure}

\end{frame}

% %------------------------------------------------
% \begin{frame}
% \frametitle{Interpretation: real-world objects}

% \begin{table}[!ht]
%   \centering
%   \begin{tabular}{l*{4}{c}p{2cm}}
%   \hline
%   \textbf{Property} & Texture & Albedo & Specular & Roughness & Best-suited techniques\\
%   \hline
%   (a) & 0.2 & 0.8 & 0.2 & 0.2 & EPS, GSL\\
%   (b) & 0.2 & 0.8 & 0.8 & 0.2 & EPS, GSL\\
%   (c) & 0.8 & 0.8, 0.2 & 0.2 & 0.8 & PMVS, GSL\\
%   (d) & 0.8 & 0.5 & 0.8 & 0.2 & PMVS\\
%   \hline
%   \end{tabular}
%   \caption{Property lists of the test objects.}
%   \label{tab:prop_list_synth_data}
% \end{table}

% \end{frame}

%------------------------------------------------
\section{Conclusions} % Sections can be created in order to organize your presentation into discrete blocks, all sections and subsections are automatically printed in the table of contents as an overview of the talk
%------------------------------------------------

\begin{frame}
\frametitle{Overview} % Table of contents slide, comment this block out to remove it
\tableofcontents[currentsection,currentsubsection, 
    hideothersubsections, 
    sectionstyle=show/shaded,] % Throughout your presentation, if you choose to use \section{} and \subsection{} commands, these will automatically be printed on this slide as an overview of your presentation
\end{frame}

\begin{frame}
\centering
Questions?
\end{frame}


% %------------------------------------------------
% \section{Reference} % Sections can be created in order to organize your presentation into discrete blocks, all sections and subsections are automatically printed in the table of contents as an overview of the talk
% %------------------------------------------------

% \begin{frame}[allowframebreaks]
%         \frametitle{References}
%         \bibliographystyle{amsalpha}
%         \bibliography{../../Thesis/biblio.bib}
% \end{frame}

%----------------------------------------------------------------------------------------

\end{document} 