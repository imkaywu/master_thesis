%% The following is a directive for TeXShop to indicate the main file
%%!TEX root = diss.tex

\chapter{Related Work}
\label{ch:RelatedWork}

We present in this chapter the related work on appearance modelling and 3D reconstruction techniques. In section \ref{sec:3DReconTech}, we provide a comprehensive review of the field of image-based 3D reconstruction based on varied visual cues. In section \ref{sec:VisProp}, we investigate the contributing factors to object appearance, which serves to choose the best-suited appearance model and reconstruction technique.

% The structure of the related work: the first subsection will focus on various kinds of 3D reconstruction techniques; the second subsection will focus on description and characterization of visual properties, which partly came from the WACV paper, but I think we need to expand on that (maybe a more quantitative way of describing these properties, for instance, BxDF for reflectance, etc, and the chapter correponding to this part of review would be derivation of algorithm-related parameters from those quantitative description.)

% \begin{itemize}
% \item 3D reconstruction techniques (use the survey I did). Instead of organize the methods based on static-dynamic techniques which is widely used in the literature, or the spatial-temporal framework, we decide to categorize the methods based on the visual cues used to reconstruct.
% 	\begin{itemize}
% 	\item Shading: shape from shading, photometric stereo
% 	\item Stereo cue: MVS, structured light
% 	\begin{itemize}
% 	\item voxel occupancy based methods
% 	\item surface evolution based methods
% 	\item region growing based methods
% 	\item depth-maps mergi{}ng based methods
% 	\end{itemize}
% 	\item Silhouette: shape from silhouette
% 	\item Texture: shape from texture variation
% 	\item Defocus: shape from defocus
% 	\end{itemize}
% \item Visual Properties and Characterization
% 	\begin{itemize}
% 	\item Reflectance
% 		\begin{itemize}
% 		\item Lightness (albedo)
% 		\item Diffusion/Specularity
% 		\item Texture:
% 		\end{itemize}
% 	\item Transparency: Ihrke's paper
% 		\begin{itemize}
% 		\item Transparent
% 		\item Translucent
% 		\item Opaque
% 		\end{itemize}
% 	\item Geometry
% 		\begin{itemize}
% 		\item microscopic: Shree Nayar's paper on physicics-based vision
% 		\item Mesoscopic/medium-scale:
% 		\item macroscopic: convexity, 
% 		\end{itemize}
% 	\end{itemize}
% \end{itemize}

%%%%%%%%%%%%%%%%%%%%%%%%%%%%%%%%%%%%%%%%%%%%%%%%%%%%%%%%%%%%%%%%%%%%%%
\section{Software}
\begin{itemize}
\item PMVS+Bundler+Poisson Recon
\item VisualSfM+CMPMVS
\item Multi-View Environment (MVE)
\item Open Multiple View Geometry (OpenMVG)
\end{itemize}
\section{3D Reconstruction Techniques}
\label{sec:3DReconTech}

Image-based 3D reconstruction attempts to recover the geometry of the object or scene from images under different viewpoints or illuminations. The goal of image-based 3D reconstruction can be described as ``given a set of images of an object or a scene, estimate the most likely 3D shape that explains those images, under the assumption of known materials, viewpoints, and lighting conditions''. This definition reveals that if those assumptions are invaild, this becomes an ill-posed problem since multiple combinations of geometry, viewpoint and illumination can produce exactly the same images~\cite{poggio1985computational}. Traditionally, the most common way of dealing with this ambiguity has been to apply smoothness heuristics and regularization techniques~\cite{poggio1985computational} to obtain reconstructions that are as smooth as possible. A drawback of this type of approach is that it typically penalizes discontinuities and sharp edges, features that are very common in real scenes.

The 3D reconstruction techniques are typically categorized as passive and active methods depending on whether the controlled illumination is required. Passive methods do not require controlled light and can work with ambient light whereas active methods require some form of temporal or spatial modulation of the illumination. We can approach the categorization based on the image cues used to reconstruct the geometry: stereo, shading, contours, texture, defocus, etc. We present different techniques based on the cues exploited in this review.

Three-dimensional model acquisition has always been one of the fundamental research topics in computer vision. Active 3D scanners are currently the dominant technology for capturing digital object models for applications. Their geometric accuracy has continually improved. But they remain expensive, and, more importantly, they suffer from a number of technical limitations. They are invasive and some materials such as hair can not be scanned. They are also not ``scalable'' to objects of different sizes, especially large ones and outdoor scenes. In comparison, passive image-based modeling from collections of images captured by handheld cameras offers several advantages. It needs only low-cost hardware, it can be applied to objects of any size, and also it preserves the appearance information from original photographs while maintaining perfect geometric alignment.



% \subsection{Silhouette Cue}

% \subsection{Texture Cue}

% \subsection{Defocus Cue}

%%%%%%%%%%%%%%%%%%%%%%%%%%%%%%%%%%%%%%%%%%%%%%%%%%%%%%%%%%%%%%%%%%%%%%
\section{Visual Properties}
\label{sec:VisProp}

TBD





\endinput
