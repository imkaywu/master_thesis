%% The following is a directive for TeXShop to indicate the main file
%%!TEX root = diss.tex

\chapter{Abstract}

Advancements in state-of-the-art 3D reconstruction algorithms have sped ahead of the development of interfaces that improve the algorithms' accessibility for application developers. In light of this discrepancy, we propose a novel abstraction framework specifically for shape capture techniques, designed to allow developers to create their own camera setups and apply the most effective method to retrieve an object's shape and appearance for their particular application.

We see that details of 3D reconstruction algorithms can be hidden by an abstraction, that uses a description based on camera setup and the characteristics of the object. We demonstrate that the description can be mapped to one or more methods to provide the user's requested result. Given that no single algorithm can work in all situations, we select a suite of three algorithms, each working in substantially different conditions, to provide an example of how the space of 3D reconstruction can be filled.

We evaluate our abstraction through a proof-of-concept implementation of the algorithm framework and a synthetic dataset where each object has been imaged with the appropriate setup for each algorithm. We demonstrate that the mapping from object characteristics to 3D shape is effective, and provides an illustration of an accessible form of 3D shape reconstruction for non-experts in computer vision, such as application developers.

% Consider placing version information if you circulate multiple drafts
%\vfill
%\begin{center}
%\begin{sf}
%\fbox{Revision: \today}
%\end{sf}
%\end{center}
