%% The following is a directive for TeXShop to indicate the main file
%%!TEX root = diss.tex

\chapter{A Mapping of 3D Reconstruction Techniques}
\label{ch:3DRecon_Mapping}
Most of the vision work focus on developing algorithmic novelties, and very few investigates the rigorous conditions under which these algorithms work. Thus this knowledge is only known empirically, without a rigorous definition of the problem or application domain or the conditions. This section build on the 3D description proposed in Chapter~\ref{ch:3DRecon_Desc}, and attempts to find out the optimal algorithms for all possible conditions.

To achieve this goal, we need a dataset to evaluate the performance of each algorithm under varied conditions, which is not the goal of most of the online dataset. To the best of our knowledge, current existing 3D benchmarks focus on one specific class of algorithms, for example, the Middlebury dataset is targeted to MVS algorithms, and the `DiLiGenT' dataset is for Photometric Stereo algorithms. This makes them suitable only to the evaluation of algorithms within the same category. There is no dataset that evaluates 3D reconstruction across differ categories, let alone one that covers a range of properties and their combinations. The reasons for the lack of such dataset is: 1). it's tedious to create a real-world dataset for a specific category of algorithm, it would be more challenging to create datasets for a range of categories with the ground truth; 2). it's practically impossible to make one property (\eg, noise level, lighting configuration, material, etc) varied while fixing the other in order to conduct a thorough evaluation.

We propose a synthetic but realistic (physically-based) dataset for evaluation of 3D reconstruction algorithms. The dataset includes a collection of images of a scene under different material or lighting conditions. The camera/projector intrinsic and extrinsic parameters are computed directly from the configurations of the synthetic setup, and the ground truth, including the 3D model point cloud and normal map, are generated directly from Blender.

\section{Synthetic setup}
We use the physical-based renderer Cycles in Blender to generate the synthetic dataset. For each technique, the configuration of the camera remains fixed. The image resolution is 1280$\times$720, with a focal length of $35mm$.

For MVS, there are five rings of cameras, of which the elevation angle is 15$^\circ$, 30$^\circ$, 45$^\circ$, 60$^\circ$, 90$^\circ$. The between-angle of two neighbouring cameras is $30^\circ$, $30^\circ$, $45^\circ$, $45^\circ$, and $360^\circ$. Thus there are in total $12+12+8+8+1=41$ cameras.

For photometric stereo, according to \cite{Berkiten:2016:ARB}, increasing the number of images is only important up to a point, the experimental results showed that most algorithms reaches to optimum when 15 images are used. To make a balance between algorithm performance and rendering time, we use 25 light sources, which are distributed on four different rings with elevation angle of $90^\circ$, $85^\circ$, $60^\circ$, and $45^\circ$. The azimuth angle between two neighbouring light sources is $45^\circ$.

For the structured light, the baseline angle between the camera and the projector is $10^\circ$, and only one camera is used, thus only a portion of the object is invisible. The resolution of the projector is $1024\times768$, thus 10 Gray code patterns are needed. To counter the effect of inter-reflection, each pattern and its inverse are projected, which makes it less sensitive to scattered light.

\section{Structure of Datasets}
Due to the number of properties and number of levels for each property, it would be unrealistic to render all the combinations of properties. For if we have $N$ properties and each is discretized into $L$ levels, the number of different combinations is $L^N$, and for each combination, there are in total $41+25+42=108$ images to render. Therefore, we take another approach: 1). first we investigate the \textit{dependency} between any two properties, if these two properties are independent, there is no need to render all their combinations whereas it's necessary to do so if they are dependent; 2). render all the dependent properties and their combinations.

\section{Selected methods}
We have selected one representative algorithm from each class of algorithms: the PMVS proposed in~\cite{furukawa2010accurate}, the example-based photometric stereo proposed in~\cite{hertzmann2005example}, and the Gray-encoded structured light technique. The current implementation of SL projects both column and row patterns, and depth values are computed using these two kinds of patterns individually. A depth consistency checking step is performed to reject erreneous triangulations.

\section{Evaluation metrics}
We use the metric proposed in \cite{seitz2006comparison} to evaluate MVS and SL algorithms. More specifically, we compute the accuracy and completeness of the reconstruction. For accuracy, the distance between the points in the reconstruction $R$ and the nearest points on ground truth $G$ is computed, and the distance $d$ such that $X\%$ of the points on $R$ are within distance $d$ of $G$ is considered as accuracy. Thus the lower the accuracy value, the better the reconstruction result. The completeness measures the fraction of points of $G$ that are within an allowable distance $d$ of $R$. Note that as the reconstruction gets better, the accuracy value goes down and the completeness value goes up. To make it more consistent to the natural context, we say the accuracy goes up when the accuracy `value' goes down.

For photometric stereo, we employ another evaluation criteria, which is based on the statistics of angular error. For each pixel, the angular error is calculated as $arccos$($n_g^T n$) in degrees, where $n_g$ and $n$ are ground truth and estimated normals respectively. In addition to the mean angular error, we also calculate the minimum, maximum, median, the first quartile, and the third quartile of angular errors for each estimated normal map.

\section{Dependency Check}
Part of the difficulty in establishing a comprehensive set of experiments for such an evaluation is the large variability of shapes and material properties. We conduct a dependency check that evaluates the performance of the algorithm by changing the values of two properties at a time while fixating the others. The goal is to identify the dependent properties so that the dimension of the problem domain could become more manageable.

\subsection{PMVS}
We evaluate the performance of PMVS in terms of accuracy and completeness under varied combinations of properties, the settings of the properties and all their combinations are listed in Table~\ref{tab:mvs_depend_check_params}.
\begin{table}[!htbp]
  \centering
  \begin{tabular}{l*{4}{c}}
  \hline
  \textbf{Property} & Texture & Albedo & Specular & Roughness\\
  \hline
  \textbf{(a)} & [0.2, 0.8] & [0.2, 0.8] & 0.0 & 0.0\\
  \textbf{(b)} & [0.2, 0.8] & 0.8 & [0.2, 0.8] & 0.0\\
  \textbf{(c)} & [0.2, 0.8] & 0.8 & 0.0 & [0.2, 0.8]\\
  \textbf{(d)} & 0.8 & [0.2, 0.8] & [0.2, 0.8] & 0.0\\
  \textbf{(e)} & 0.8 & [0.2, 0.8] & 0.0 & [0.2, 0.8]\\
  \textbf{(f)} & 0.8 & 0.8 & [0.2, 0.8] & [0.2, 0.8]\\
  \hline
  \end{tabular}
  \caption{Parameter settings of dependency check of MVS.}
  \label{tab:mvs_depend_check_params}
\end{table}

\begin{figure}[!htbp]
\begin{tabular}{cc}
\includegraphics[width=0.5\textwidth]{mapping/depend_check/mvs_tex_alb}&
\includegraphics[width=0.5\textwidth]{mapping/depend_check/mvs_tex_spec}\\
(a) & (b)\\
\includegraphics[width=0.5\textwidth]{mapping/depend_check/mvs_tex_rough}&
\includegraphics[width=0.5\textwidth]{mapping/depend_check/mvs_alb_spec}\\
(c) & (d)\\
\includegraphics[width=0.5\textwidth]{mapping/depend_check/mvs_alb_rough}&
\includegraphics[width=0.5\textwidth]{mapping/depend_check/mvs_spec_rough}\\
(e) & (f)\\
\end{tabular}
\caption{Performance of MVS with varied properties}
\label{fig:mvs_depend_check}
\end{figure}

Now let's investigate the correlation between each property or each pair of properties and the chosen metrics, as shown in Table~

\subsubsection{Single property}
\textbf{Texture} - \textit{(a), (b), (c)}. As the texture level increases, the accuracy and completeness both increase. Thus texture has a positive correlation with the accuracy and completeness of the reconstruction.

\textbf{Albedo} - \textit{(a), (d), (e)}. we can tell that albedo doesn't have an effect when texture or roughness changes. But when the specular level changes, albedo has an effect on both accuracy and completeness of the reconstruction.

\textbf{Specular} - \textit{(b), (d), (f)}. As the specular level goes up, the accuracy and completeness of the reconstruction gets worse. Thus specular has an negative correlation with the accuracy and completeness of the reconstruction.

\textbf{Roughness} - \textit{(c), (e), (f)}. We can see that roughness doesn't have a influence on the reconstruction.

\subsubsection{Pairwise properties}
Now let's investigate the influence of pairwise properties. We observed that (b), and (d) has significant changes in terms of accuracy and completeness when one property changes while the other is fixed.

\textbf{(b) Texture and Specularity} 
For a fixed texture, as the specularity goes up, the accuracy and the completeness goes up, which is consistent to previous observations. Besides, for a lower value texture, the effect of specular is more substantial than that for a higher value texture.

\textbf{(d) Albedo and Specularity} 
For a fixed albedo, as the specular goes up, the accuracy and completeness both goes down, which is consistent to previous observations. However, the effect of specular is more substantial for a lower value albedo than that for a higher value albedo.

% \textbf{(f) Specularity and Roughness} 
% Justification for roughness: if the specular surface is smooth, the reconstruction is horrible, but if roughness is present, however small the roughness is, the error caused by specular can be effectively mitigated.

\textbf{Conclusion} the properties that have an effect on the MVS are: texture, albedo, and specularity, as shown in Table~\ref{tab:mvs_depend_prop}. Thus, we will only consider these three properties for all forthcoming discussion of MVS.
\begin{table}[!htbp]
  \centering
  \begin{tabular}{l*{4}{c}}
  \hline
  \textbf{Metric} & Texture & Albedo & Specular & Roughness\\
  \hline
  Accuracy & \ding{55} & \checkmark & \checkmark & \ding{55}\\
  Completeness & \checkmark & \checkmark & \checkmark & \ding{55}\\
  \hline
  \end{tabular}
  \caption{The correlation between each property and the metrics \textit{accuracy} and \textit{completeness}.}
  \label{tab:mvs_depend_prop}
\end{table}

% \begin{table}[!htbp]
%   % \centering
%   \begin{tabular}{*{4}{c}r||*{4}{c}r||*{4}{c}r}
%   \hline
%   T & A & S & R & RS & T & A & S & R & RS & T & A & S & R & RS\\
%   \hline
%   0.2 & 0.2 & 0.2 & 0.0 & \ding{55} & 0.5 & 0.2 & 0.2 & 0.0 & \ding{55} & 0.8 & 0.2 & 0.2 & 0.0 & \ding{55}\\
%   0.2 & 0.2 & 0.5 & 0.0 & \ding{55} & 0.5 & 0.2 & 0.5 & 0.0 & \ding{55} & 0.8 & 0.2 & 0.5 & 0.0 & \ding{55}\\
%   0.2 & 0.2 & 0.8 & 0.0 & \ding{55} & 0.5 & 0.2 & 0.8 & 0.0 & \ding{55} & 0.8 & 0.2 & 0.8 & 0.0 & \ding{55}\\
%   0.2 & 0.5 & 0.2 & 0.0 & \ding{55} & 0.5 & 0.5 & 0.2 & 0.0 & \ding{55} & 0.8 & 0.5 & 0.2 & 0.0 & \ding{55}\\
%   0.2 & 0.5 & 0.5 & 0.0 & \ding{55} & 0.5 & 0.5 & 0.5 & 0.0 & \ding{55} & 0.8 & 0.5 & 0.5 & 0.0 & \ding{55}\\
%   0.2 & 0.5 & 0.8 & 0.0 & \ding{55} & 0.5 & 0.5 & 0.8 & 0.0 & \ding{55} & 0.8 & 0.5 & 0.8 & 0.0 & \ding{55}\\
%   0.2 & 0.8 & 0.2 & 0.0 & \ding{55} & 0.5 & 0.8 & 0.2 & 0.0 & \ding{55} & 0.8 & 0.8 & 0.2 & 0.0 & \ding{55}\\
%   0.2 & 0.8 & 0.5 & 0.0 & \ding{55} & 0.5 & 0.8 & 0.5 & 0.0 & \ding{55} & 0.8 & 0.8 & 0.5 & 0.0 & \ding{55}\\
%   0.2 & 0.8 & 0.8 & 0.0 & \ding{55} & 0.5 & 0.8 & 0.8 & 0.0 & \ding{55} & 0.8 & 0.8 & 0.8 & 0.0 & \ding{55}\\
%   \hline
%   \end{tabular}
%   \caption{Mapping from the problem conditions to PMVS}
% \end{table}

\subsection{Example-based PS}
We evaluate the performance of example-based PS in terms of angular difference under varied combinations of properties, The statistical measures that we used include median, mean, first and third quartile of the angular difference. We investigate two properties at a time. The settings of the properties and all their combinations are listed in Table~\ref{tab:ps_depend_check_params}.

\begin{table}[!htbp]
  \centering
  \begin{tabular}{l*{4}{c}}
  \hline
  \textbf{Property} & Texture & Albedo & Specular & Roughness\\
  \hline
  \textbf{(a)} & [0.2, 0.8] & [0.2, 0.8] & 0.0 & 0.0\\
  \textbf{(b)} & [0.2, 0.8] & 0.8 & [0.2, 0.8] & 0.2\\
  \textbf{(c)} & [0.2, 0.8] & 0.8 & 0.0 & [0.2, 0.8]\\
  \textbf{(d)} & 0.0 & [0.2, 0.8] & [0.2, 0.8] & 0.2\\
  \textbf{(e)} & 0.0 & [0.2, 0.8] & 0.0 & [0.2, 0.8]\\
  \textbf{(f)} & 0.0 & 0.8 & [0.2, 0.8] & [0.2, 0.8]\\
  \hline
  \end{tabular}
  \caption{Parameter settings of dependency check of example-based PS.}
  \label{tab:ps_depend_check_params}
\end{table}

\begin{figure}[!htbp]
\begin{tabular}{cc}
\includegraphics[width=0.5\textwidth]{mapping/depend_check/ps_tex_alb}&
\includegraphics[width=0.5\textwidth]{mapping/depend_check/ps_tex_spec}\\
(a) & (b)\\
\includegraphics[width=0.5\textwidth]{mapping/depend_check/ps_tex_rough}&
\includegraphics[width=0.5\textwidth]{mapping/depend_check/ps_alb_spec}\\
(c) & (d)\\
\includegraphics[width=0.5\textwidth]{mapping/depend_check/ps_alb_rough}&
\includegraphics[width=0.5\textwidth]{mapping/depend_check/ps_spec_rough}\\
(e) & (f)\\
\end{tabular}
\caption{Performance of PS with varied properties}
\label{fig:ps_depend_check}
\end{figure}

\subsection{Single property}
\textbf{Texture} - \textit{(a), (b), (c)}. As the texture level increases, all statistic measures of the angular difference remain almost the same. Thus texture doesn't affect the reconstruction of the chosen PS algorithm.

\textbf{Albedo} - \textit{(a), (d), (e)}. As the albedo level increases, all statistic measures of the angular difference decreases. Thus the albedo has a positive correlation to the reconstruction.

\textbf{Specular} - \textit{(b), (d), (f)}. As the specular level goes up, all statistic measures of the angular difference increases. Thus specular has an negative impact on the reconstruction.

\textbf{Roughness} - \textit{(c), (e), (f)}. The effect of roughness is a bit complicated. Generally, it will improve the reconstruction as the roughness goes higher. However, as shown in Figure~\ref{fig:ps_depend_check} (f), the roughness will cause worse reconstruction for medium high value, which will be discussed more later.

\subsection{Pairwise properties}
All the pairwise relations are straightforward except for (f).

\textbf{(f) Specularity and Roughness} 
For a fixed specularity, if the specularity is lower, the effect of roughness is less noticeable, whereas if the specularity is higher, the effect of roughness becomes more substantial. We've also noticed a `peculiar' case when roughness is 0.5, it makes the reconstruction worse, which is counter-intuitive. However, we argue that it's because the roughness effect is not strong enough to cancel out the specularity, thus causing a much larger area of `blurred' specularity, which makes the reconstruction worse. This effect is also demonstrated in the training stage, see Figure~\ref{fig:ps_outlier} for some visual examples.
\begin{figure}[h!]
\centering
\begin{tabular}{ccc}
  \includegraphics[width=0.35\textwidth]{mapping/ps_rough/00020202_0001}&
  \includegraphics[width=0.35\textwidth]{mapping/ps_rough/00020202_normal}&
  \includegraphics[width=0.30\textwidth]{mapping/ps_rough/00020202_boxplot}\\
  (a).020202 & (b). normal map & (c). angle diff distribution\\
  \includegraphics[width=0.35\textwidth]{mapping/ps_rough/00020205_0001}&
  \includegraphics[width=0.35\textwidth]{mapping/ps_rough/00020205_normal}&
  \includegraphics[width=0.30\textwidth]{mapping/ps_rough/00020205_boxplot}\\
  (d).020205 & (e). normal map & (f). angle diff distribution\\
  \includegraphics[width=0.35\textwidth]{mapping/ps_rough/00080205_0001}&
  \includegraphics[width=0.35\textwidth]{mapping/ps_rough/00080205_normal}&
  \includegraphics[width=0.30\textwidth]{mapping/ps_rough/00080205_boxplot}\\
  (g).080205 & (h). normal map & (i). angle diff distribution\\
  \includegraphics[width=0.35\textwidth]{mapping/ps_rough/00080805_0001}&
  \includegraphics[width=0.35\textwidth]{mapping/ps_rough/00080805_normal}&
  \includegraphics[width=0.30\textwidth]{mapping/ps_rough/00080805_boxplot}\\
  (j).080805 & (k). normal map & (l). angle diff distribution\\
\end{tabular}
\caption{The `peculiar' effect of roughness on PS. The order of the property is: albedo, specular, and roughness, thus 080205 means albedo: 0.8, specular: 0.2, and roughness: 0.5}
\label{fig:ps_outlier}
\end{figure}

\textbf{Conclusion} the properties that have an effect on the PS are: albedo, specularity, and roughness, as shown in Table~\ref{tab:ps_depend_prop}. Therefore, we will only consider these three properties for all forthcoming discussion of PS.
\begin{table}[!htbp]
  \centering
  \begin{tabular}{l*{5}{c}}
  \hline
  \textbf{Metric} & Texture & Albedo & Specular & Roughness\\
  \hline
  Angle difference & \ding{55} & \checkmark & \checkmark & \checkmark\\
  \hline
  \end{tabular}
  \caption{The correlation between each property and the metric \textit{angular difference}.}
  \label{tab:ps_depend_prop}
\end{table}

\subsection{Gray-code SL}
We evaluate the performance of Gray-code SL in terms of accuracy and completeness under varied combination of properties, the settings of the properties and all their combinations are listed in Table~\ref{tab:sl_depend_check_params}.
\begin{table}[!htbp]
  \centering
  \begin{tabular}{l*{4}{c}}
  \hline
  \textbf{Property} & Texture coverage & Albedo & Specular/Diffuse ratio & Roughness\\
  \hline
  \textbf{(a)} & [0.2, 0.8] & [0.2, 0.8] & 0.0 & 0.0\\
  \textbf{(b)} & [0.2, 0.8] & 0.8 & [0.2, 0.8] & 0.0\\
  \textbf{(c)} & [0.2, 0.8] & 0.8 & 0.0 & [0.2, 0.8]\\
  \textbf{(d)} & 0.0 & [0.2, 0.8] & [0.2, 0.8] & 0.0\\
  \textbf{(e)} & 0.0 & [0.2, 0.8] & 0.0 & [0.2, 0.8]\\
  \textbf{(f)} & 0.0 & 0.8 & [0.2, 0.8] & [0.2, 0.8]\\
  \hline
  \end{tabular}
  \caption{Parameter settings of dependency check of Gray code SL.}
  \label{fig:sl_depend_check_params}
\end{table}

\begin{figure}[!htbp]
\begin{tabular}{cc}
\includegraphics[width=0.5\textwidth]{mapping/depend_check/sl_tex_alb}&
\includegraphics[width=0.5\textwidth]{mapping/depend_check/sl_tex_spec}\\
(a) & (b)\\
\includegraphics[width=0.5\textwidth]{mapping/depend_check/sl_tex_rough}&
\includegraphics[width=0.5\textwidth]{mapping/depend_check/sl_alb_spec}\\
(c) & (d)\\
\includegraphics[width=0.5\textwidth]{mapping/depend_check/sl_alb_rough}&
\includegraphics[width=0.5\textwidth]{mapping/depend_check/sl_spec_rough}\\
(e) & (f)\\
\end{tabular}
\caption{Performance of SL with varied properties}
\label{fig:sl_depend_check}
\end{figure}

\subsection{Single property}
A depth check step is performed to remove erroneous depth, thus the accuracy remain almost constant across all cases.

\textbf{Texture} - \textit{(a), (b), (c)}. As the texture level increases, the accuracy and completeness remain remain almost constant. Thus texture doesn't affect the reconstruction of the chosen SL algorithm.

\textbf{Albedo} - \textit{(a), (d), (e)}. As the albedo level increases, the accuracy remain almost constant while the completeness increases. Thus albedo has a positive correlation with completeness.

\textbf{Specular} - \textit{(b), (d), (f)}. As the specular level goes up, the accuracy remain almost constant while the completeness of the reconstruction decreases. Thus specular has an negative correlation with the completeness of the reconstruction.

\textbf{Roughness} - \textit{(c), (e), (f)}. As the roughness level increases, the accuracy remain almost constant while the completeness increases. Thus roughness has a positive correlation with completeness.

\subsection{Pairwise properties}
\textbf{(d) Albedo and Specularity} 
For a fixed albedo, the completeness goes down as the specularity goes up for low albedo surface, this effect becomes less substantial when the albedo increases. Thus we conclude that the effect of specular is most significant when the albedo is low.

\textbf{Conclusion} the properties that have an effect on the SL are: texture, albedo, specularity, as shown in Table. Therefore, we will only consider these three properties for all forthcoming discussion of SL.
\begin{table}[!htbp]
  \centering
  \begin{tabular}{l*{4}{c}}
  \hline
  \textbf{Metric} & Texture & Albedo & Specular & Roughness\\
  \hline
  Accuracy & \ding{55} & \ding{55} & \ding{55} & \ding{55}\\
  Completeness & \ding{55} & \checkmark & \checkmark & \checkmark\\
  \hline
  \end{tabular}
  \caption{The correlation between each property and the metrics \textit{accuracy} and \textit{completeness}.}
  \label{tab:sl_depend_prop}
\end{table}


\section{Training and mapping}
For each technique, we generate the synthetic dataset using only the dependent properties, thus there are $L\times L\times L$ different combinations for each technique, where $L$ is the number of levels for each property.

\subsection{PMVS}
The performance of PMVS under difference combinations of properties is shown in Figure~\ref{fig:mvs_training}. The conditions that PMVS works well is listed in Table~\ref{fig:mvs_training_result}.
\begin{figure}[!htbp]
\begin{tabular}{ccc}
\includegraphics[width=0.33\textwidth]{mapping/training/mvs_train_spec_02}&
\includegraphics[width=0.33\textwidth]{mapping/training/mvs_train_spec_05}&
\includegraphics[width=0.33\textwidth]{mapping/training/mvs_train_spec_08}\\
(a) & (b) & (c)\\
\includegraphics[width=0.33\textwidth]{mapping/training/mvs_train_tex_02}&
\includegraphics[width=0.33\textwidth]{mapping/training/mvs_train_tex_05}&
\includegraphics[width=0.33\textwidth]{mapping/training/mvs_train_tex_08}\\
(d) & (e) & (f)\\
\includegraphics[width=0.33\textwidth]{mapping/training/mvs_train_alb_02}&
\includegraphics[width=0.33\textwidth]{mapping/training/mvs_train_alb_05}&
\includegraphics[width=0.33\textwidth]{mapping/training/mvs_train_alb_08}\\
(g) & (h) & (i)\\
\end{tabular}
\caption{Performance of MVS with varied properties.}
\label{fig:mvs_training}
\end{figure}

\begin{table}[!htbp]
  \centering
  \begin{tabular}{l*{4}{c}}
  \hline
  \textbf{Metric} & Texture & Albedo & Specular & Roughness\\
  \hline
  Accuracy \&  & 0.5 & 0.5 & 0.2 & -\\
  Completeness & 0.5 & 0.8 & 0.2 & -\\
               & 0.8 & 0.2 & 0.2 & -\\
               & 0.8 & 0.5 & 0.2 & -\\
               & 0.8 & 0.8 & 0.2 & -\\
               & 0.5 & 0.8 & 0.5 & - \\
               & 0.5 & 0.8 & 0.5 & -\\
               & 0.8 & 0.5 & 0.5 & -\\
               & 0.8 & 0.8 & 0.5 & -\\
               & 0.8 & 0.5 & 0.8 & -\\
               & 0.8 & 0.8 & 0.8 & -\\
  \hline
  \end{tabular}
  \caption{The conditions under which PMVS works well in terms of the two metrics \textit{accuracy} and \textit{completeness}.}
  \label{tab:mvs_traing_result}
\end{table}

\subsection{Example-based PS}
The performance of example-based PS under difference combinations of properties is shown in Figure~\ref{fig:ps_training}. The conditions that example-based PS works well is listed in Table~\ref{fig:ps_training_result}.
\begin{figure}[!htbp]
\begin{tabular}{ccc}
\includegraphics[width=0.33\textwidth]{mapping/training/ps_rough_02}&
\includegraphics[width=0.33\textwidth]{mapping/training/ps_rough_05}&
\includegraphics[width=0.33\textwidth]{mapping/training/ps_rough_08}\\
(a) & (b) & (c)\\
\includegraphics[width=0.33\textwidth]{mapping/training/ps_alb_02}&
\includegraphics[width=0.33\textwidth]{mapping/training/ps_alb_05}&
\includegraphics[width=0.33\textwidth]{mapping/training/ps_alb_08}\\
(d) & (e) & (f)\\
\includegraphics[width=0.33\textwidth]{mapping/training/ps_spec_02}&
\includegraphics[width=0.33\textwidth]{mapping/training/ps_spec_05}&
\includegraphics[width=0.33\textwidth]{mapping/training/ps_spec_08}\\
(g) & (h) & (i)\\
\end{tabular}
\caption{Performance of PS with varied properties.}
\label{fig:ps_training}
\end{figure}

\begin{table}[!htbp]
  \centering
  \begin{tabular}{l*{4}{c}}
  \hline
  \textbf{Metric} & Texture & Albedo & Specular & Roughness\\
  \hline
  Angle difference & - & 0.2 & 0.2 & 0.8\\
                   & - & 0.2 & 0.5 & 0.8\\
                   & - & 0.2 & 0.8 & 0.8\\
                   & - & 0.5 & 0.2 & 0.8\\
                   & - & 0.5 & 0.5 & 0.8\\
                   & - & 0.5 & 0.8 & 0.8\\
                   & - & 0.8 & 0.2 & 0.2\\
                   & - & 0.8 & 0.2 & 0.8\\
                   & - & 0.8 & 0.5 & 0.2\\
                   & - & 0.8 & 0.5 & 0.8\\
                   & - & 0.8 & 0.8 & 0.2\\
                   & - & 0.8 & 0.8 & 0.8\\
  \hline
  \end{tabular}
  \caption{The conditions under which example-based PS works well in terms of the two metric \textit{angular difference}.}
  \label{tab:ps_training_result}
\end{table}

\subsection{Gray-code SL}
The performance of Gray code SL under difference combinations of properties is shown in Figure~\ref{fig:sl_training}. The conditions that PMVS works well is listed in Table~\ref{fig:sl_training_result}.
\begin{figure}[!htbp]
\begin{tabular}{ccc}
\includegraphics[width=0.33\textwidth]{mapping/training/sl_train_rough_02}&
\includegraphics[width=0.33\textwidth]{mapping/training/sl_train_rough_05}&
\includegraphics[width=0.33\textwidth]{mapping/training/sl_train_rough_08}\\
(a) & (b) & (c)\\
\includegraphics[width=0.33\textwidth]{mapping/training/sl_train_alb_02}&
\includegraphics[width=0.33\textwidth]{mapping/training/sl_train_alb_05}&
\includegraphics[width=0.33\textwidth]{mapping/training/sl_train_alb_08}\\
(d) & (e) & (f)\\
\includegraphics[width=0.33\textwidth]{mapping/training/sl_train_spec_02}&
\includegraphics[width=0.33\textwidth]{mapping/training/sl_train_spec_05}&
\includegraphics[width=0.33\textwidth]{mapping/training/sl_train_spec_08}\\
(g) & (h) & (i)\\
\end{tabular}
\caption{Performance of SL with varied properties.}
\label{fig:sl_training}
\end{figure}

\begin{table}[!htbp]
  \centering
  \begin{tabular}{l*{4}{c}}
  \hline
  \textbf{Metric} & Texture & Albedo & Specular & Roughness\\
  \hline
  Accuracy \&  & - & 0.5 & 0.2 & 0.2\\
  Completeness & - & 0.5 & 0.5 & 0.2\\
               & - & 0.5 & 0.8 & 0.2\\
               & - & 0.8 & 0.2 & 0.2\\
               & - & 0.8 & 0.5 & 0.2\\
               & - & 0.8 & 0.8 & 0.2\\
               & - & 0.5 & 0.2 & 0.5\\
               & - & 0.5 & 0.5 & 0.5\\
               & - & 0.5 & 0.8 & 0.5\\
               & - & 0.8 & 0.2 & 0.5\\
               & - & 0.8 & 0.5 & 0.5\\
               & - & 0.8 & 0.8 & 0.5\\
               & - & 0.2 & 0.2 & 0.8\\
               & - & 0.2 & 0.5 & 0.8\\
               & - & 0.2 & 0.8 & 0.8\\
               & - & 0.5 & 0.2 & 0.8\\
               & - & 0.5 & 0.5 & 0.8\\
               & - & 0.5 & 0.8 & 0.8\\
               & - & 0.8 & 0.2 & 0.8\\
               & - & 0.8 & 0.5 & 0.8\\
               & - & 0.8 & 0.8 & 0.8\\
  \hline
  \end{tabular}
  \caption{The conditions under which Gray code SL works well in terms of the two metrics \textit{accuracy} and \textit{completeness}.}
  \label{tab:sl_traing_result}
\end{table}

% \begin{figure}[!htbp]
% \includegraphics[width=\textwidth]{mapping/training}
% \caption{Performance of MVS, SL and PS with varied properties. Each each column, we fix one property while changing the others, thus the second and the third columns are essentially the same as the first column, they are just different point of views of looking at those relations. Each line/boxplot represents a different combinations of property values: 0202, 0205, 0208, 0502, ..., 0808. Beware that we consider \{tex, alb, spec\} for MVS and SL, and \{alb, spec, rough\} for SL.}
% \label{fig:training}
% \end{figure}

% \section{Mapping of 3D Reconstruction}
% From the training results, we can derive a mapping between problem conditions and optimal algorithms, as shown in Table~\ref{tab:mapping}.
% \begin{table}[!htbp]
%   \centering
%   \begin{tabular}{*{7}{c}}
%   \hline
%   Texture & Albedo & Specular & Roughness & Accuracy & Completeness & Ang Diff\\
%   \hline
%   0.2 & 0.2 & 0.2 & \\
%   0.2 & 0.2 & 0.5 & \\
%   0.2 & 0.2 & 0.8 & \\
%   0.2 & 0.5 & 0.2 & \\
%   0.2 & 0.5 & 0.5 & \\
%   0.2 & 0.5 & 0.8 & \\
%   0.2 & 0.8 & 0.2 & \\
%   0.2 & 0.8 & 0.5 & \\
%   0.2 & 0.8 & 0.8 & \\
%   0.5 & 0.2 & 0.2 & \\
%   0.5 & 0.2 & 0.5 & \\
%   0.5 & 0.2 & 0.8 & \\
%   0.5 & 0.5 & 0.2 & \\
%   0.5 & 0.5 & 0.5 & \\
%   0.5 & 0.5 & 0.8 & \\
%   0.5 & 0.8 & 0.2 & \\
%   0.5 & 0.8 & 0.5 & \\
%   0.5 & 0.8 & 0.8 & \\
%   0.8 & 0.2 & 0.2 & \\
%   0.8 & 0.2 & 0.5 & \\
%   0.8 & 0.2 & 0.8 & \\
%   0.8 & 0.5 & 0.2 & \\
%   0.8 & 0.5 & 0.5 & \\
%   0.8 & 0.5 & 0.8 & \\
%   0.8 & 0.8 & 0.2 & \\
%   0.8 & 0.8 & 0.5 & \\
%   0.8 & 0.8 & 0.8 & \\
%   \hline
%   \end{tabular}
%   \caption{The mapping from property conditions to algorithms.}
%   \label{tab:mapping}
% \end{table}