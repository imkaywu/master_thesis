%% The following is a directive for TeXShop to indicate the main file
%%!TEX root = diss.tex

\chapter{Conclusions}
\label{sec:conclusion}

\section{Summary}

\section{Future directions}

\subsection{Geometric Model}
The current geometric model fails to capture the complexity of real world objects and focuses mainly on visual properties. Thus, one of the future directions is to develop intuitive geometric models to better describe complex objects.

\subsection{Property Parameters}
We have used a ``try-and-see'' approach to obtain the property settings of an object, which is based on user judgement and is thus not very rigorous and tends to be tedious. We can use machine learning techniques to obtain visual and geometric information.

\subsection{Metrics}
We have utilized three metrics: accuracy, completeness and angular error. However, there are other measures worth investigating, such as colour accuracy, `ghost' reconstruction error, and so on. Additional metrics such as these can extend the application of our framework, providing more options for developers to choose from.

\subsection{Mapping Construction}
The construction of mapping requires an evaluation of the corresponding algorithm for pairwise properties, which tends not to scale well with respect to the number of properties. Therefore, we need better ways to discover the dependent relation bewteen any two properties.

\subsection{Interpreter}
Currently, implementation of the proof-of-concept interpreter is simplistic and does not fully take advantage of the information we have obtained from the mapping construction process. Therefore, we should develop a more sophisticated interpreter that is more powerful and offers more flexibility.

\section{Concluding remarks}
