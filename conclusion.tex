%% The following is a directive for TeXShop to indicate the main file
%%!TEX root = diss.tex

\chapter{Conclusions}
\label{ch:conclusion}
With the increased sophistication of vision algorithms, comes the increased barriers of applying these algorithms in real world applications. Just as personal computers did not become mainstream until the graphical user interface, computer vision algorithms will be less likely to reach to masses until creative minds from all discplines can take advantage of these amazing technologies without needing expertise knowledge. To address this challenge, we proposed a three-layer interface, consisting of description, interpreter, and algorithms, to allow users with no vision background to take advantage of these vision algorithms. We review each of the chapters in details below.

\subsubsection{A problem space of 3D reconstruction}
In this section, we presented a well-defined problem space for 3D reconstruction algorithms based on the conditions surrounding the problem. It is an $N-$dimensional space, the axes of which are visual and geometric properties of objects. This allows users to think of an algorithm as a ``pointer'' to a sub-volumn within the space that it can reliably work under. This object centric problem space allows application developers to have access to advanced vision techniques without requiring sophisticated vision knowledge of the underlying algorithms.

\subsubsection{A description of 3D reconstruction}
After proposing a problem space that allows to associate algorithms to problem conditions, we proceeded to discuss a way of describing the problem condition of a 3D reconstruction problem, which provides an abstraction layer above the underlying vision algorithms. The description consists of a model and corresponding representations. The model selects characteristic visual and geometric properties of objects as components and the representation uses the key aspect of a property to quantify the strength of the specific property. The proposed description provides a formal and definitive way of describing the problem condition. We further discuss the ease of quantifying the properties, and provide concrete examples to demonstrate the process. The performance of the algorithm can be evaluated given a well defined description, allowing a better understanding of the working conditions of the specific algorithm.

\subsubsection{A mapping of 3D reconstruction}
Once we obtained the means of describing the problem conditions, we set out to investigate the working conditions surrounding each algorithm. First, we investigate the impact of pairwise properties on the performance of three algorithms across categories, from which the \textit{effective properties} are determined. Next, we evaluate the selected algorithms under different problem conditions consisting only of these \textit{effective properties}. This allows us to derive a mapping from problem conditions to algorithms by comparing the performance to that of the baseline methods. This information provides a deeper understanding of performance of algorithms and potentially insights into how they may be improved.

\subsubsection{An interpretation of 3D reconstruction}
Lastly, we demonstrated the use of the interface by a proof of concept interpreter, which returns a successful reconstruction result given a valid description of an object's problem condition. The interpreter is the intermediate layer of the interface, which receives a user-specified description and invokes one of the underlying algorithms for reconstruction. We are interested to see how the interface would perform given accurate, less accurate and incorrect descriptions. First, we proposed a proof of concept interpreter that takes advantage of the discovered mapping and additional constraints. Next, we presented the real-world and synthetic datasets for evaluation. Lastly, we demonstrated the performance of the interpreter by providing accurate and inaccurate descriptions. The performance of the interface echos with the statement of the thesis that a description-based interface for 3D reconstruction without knowing algorithm detail is achievable.\\

We have discovered that the interpreter can produce a successful reconstruction result reliably given accurate description while fails to achieve successful results given less accurate descriptions in some cases, and is much more likely to give poor reconstruction results given completely incorrect descriptions. Thus, we conclude that it is achieveable to design a description-based interface that leads to a successful reconstruction given a correct description of problem condition. Though, the interface, to some extent, is tolerant to the inaccurate description of problem condition. Nonetheless, the correct interpretation of problem condition is key to a successful reconstruction. The main contribution of this thesis is the development and application of an interface to 3D reconstruction problem. The significance of this contribution is that: 1) few algorithms can work for a diverse categories of objects. The interface, to some extent, can cover a wider range of object categories by incorporating multiple algorithms; 2) a description of object problem condition is provided to hide the algorithmic details, thus understanding of the algorithm, or conditions of applying algorithms are not a prerequisite.

Extending beyong 3D reconstruction, our proposed three-layer interface to general vision problems can be applied to allow other vision tasks be more accessible to application developers by providing a description to vision problems which allows the specifications of problem conditions. In order to provide such accessibility, a description of a well defined problem space, and ways to discover the optimal problem conditions must be addressed. This is a non-trivial task to provide such an abstraction as it requires an understanding of the field and an ability to abstract away algorithmic complexity. Further, the construction of mapping requires more than just appropriate datasets, but also means of avoiding property scaling issue, which would make the problem space too vast to handle. In conclusion, this thesis addresses the accessibility of one of vision topics, 3D reconstruction problem, which provides a novel perspective of approaching 3D vision problem without delving into algorithm details, and establishes a general framework of description-based interface that could be extended to other vision problems.

\section{Future directions}
3D reconstruction has been one of the most important topics in vision for decades with a range of applications. This thesis focuses on the accessibility of these algorithms instead of developing algorithm novelties. We make several assumptions and simplifications in this thesis, and thus opens up some potential future directions that can improve the work completed in this thesis.

\subsubsection{Geometric model}
The current model fails to capture the geometric complexity of real world objects and focuses mainly on visual properties. Target objects with complex geometric properties, such as severe concavity, occlusion, depth (dis-)continuity, and so on are not captured in the dataset. The issue of incorporating these geometric properties is due to the dilemma of mathematical and semantic representation: mathematical representations are easier to estimate but difficult for human interpretation while semantic representations are easier for humans to understand but challenging to be modelled mathematically. For instance, surface concavity can be mathematically modelled by curvature, but this is non-trivial to estimate the parameter of curvature and translate the numeric value to semantically meaningful terms. Concavity can also be represented as a continuum with two extremities: convexity and concavity, which is intuitive to humans. However, it is hard to interpolate to obtain any semantic terms in between, or get it translated to the mathematical counterpart. Thus, one of the research directions is to develop intuitive and semantically meaningful geometric model to better describe objects with complex shapes. This could siginifically expand the problem space and allows a much more powerful interface that target a broader categories of objects.

\subsubsection{Property estimation}
We have used a approach of visual inspection and ``trial-and-see'' to simulate and estimate the property settings of an object, which is based on user input and thus is relatively subjective, tedious, less rigorous and prone to error. A more robust approach is to utilize machine learning techniques to obtain visual and geometric information directly from images.

\subsubsection{Quantitative measure}
We have utilized three metrics: accuracy, completeness and angular error. However, there are other measures worth investigating. For instance, colour accuracy that measures the accuracy of the colour information of the reconstructed model, and `ghost' error, which measures the amount of erroneously reconstructed object or scene, and so on. We provide the users with more options and controls over the reconstruction result that suits their needs by providing additional quantitative measures.

\subsubsection{Mapping construction}
The key component of the interpreter is the mapping from problem conditions to algorithms. This mapping information gives the interpreter information regarding what algorithms work reliably under a given problem condition. Every algorithm that is to be integrated into the interpreter needs to be evaluated under a variety of problem conditions. However, this process might suffer from property scaling issues. For instance, if the number of \textit{effective properties} is $N$ and each property has $L$ levels, the number of different combinations is $L^N$, which increases expoentionally as the number of properties increases. Therefore, we need better ways to discover this mapping relating problem space and algorithms. Otherwise, the mapping construction suffers from what we call \textit{property scaling issue}.

\subsubsection{Interpreter}
Interpreter is the intermediate layer of the three-layer interface, which is responsible for receive user-specified description and invokes one of the successful underlying algorithms. Currently, the implementation of the proof of concept interpreter is simplistic and does not fully take advantage of the information we have obtained from the mapping construction process. Therefore, we should develop a more sophisticated interpreter that is more powerful and offers more flexibility and options to the users.

\section{Closing words}
Nowadays, computer vision has been playing an increasingly important role in a variety of fields. However, it also requires application developers increasing expertise to apply these technologies to a specific application domain. We hope that the efforts in the vision community can be directed to not only develop more advanced algorithms, but also easier access to these algorithms as well. The development of such a system across vision problems will require significant efforts in the form of detailed problem space, comprehensive description, sophisticated methods of interpretation, and so on. This is the work of the entire community of vision researchers, and could lead to advancements in the field similar to those seen in Computer Graphics following the creation of OpenGL. We hope that this thesis provides a starting point for such a significant undertaking, highlighting our initial approach to the problem, and the challenges that researchers who choose to follow may face.