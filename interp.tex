%% The following is a directive for TeXShop to indicate the main file
%%!TEX root = diss.tex

\chapter{An Interpretation of 3D Reconstruction}
\label{ch:3DRecon_Interp}
In Chapter~\ref{ch:3DRecon_Mapping}, we have established a mapping from a well defined problem space to a suite of algorithms through evaluating the performance under synthetic situations. However, the claim that this mapping would help the users obtain a satisfactory reconstruction result given the correct problem conditions is still unclear. Thus a thorough evaluation is needed to validate the proposed framework.

However, such an evaluation faces several challenges: 1). the derived mapping pose very few constraints on the types of material and geometry, thus the evaluation should target a vast amount of objects to reach a solid conclusion, which obvious is not a practical approach; 2). Section~\ref{sec:interp_eval_methodology} gives a the roadmap of our evaluation which is centered around two key evaluation questions: extensiveness of the derived mapping, and usefulness of the algorithm-free framework. Section~\ref{sec:interp_extend} demonstrate under which cases the derived mapping can be safely applied to other objects. Section~\ref{sec:interp_useful} presents real-world use cases of the framework, where a satisfactory reconstruction result is return given the correct description of object.

% In order to validate the 3D reconstruction mapping derived from Chapter~\ref{ch:3DRecon_Mapping}, evaluation of the object centric model into appropriate solutions must be shown. Our interpreter is based on the direct evaluation of the performance of each 3D reconstruction algorithm under different conditions presented in Chapter~\ref{ch:3DRecon_Mapping}. From this analysis of how algorithms perform on objects which have different visual and geometric properties, an algorithm(s) can be definitively chosen based on which performed best on the training images.

% Although only three algorithms selected, all of which are the top performers in the corresponding field, thus are sufficient to demonstrate the framework's ability to translate the descriptive model into a reconstruction. Furthermore, the integration of a new algorithm requires the same process preseneted in Chapter~\ref{ch:3DRecon_Mapping}, allowing researchers to contribute novel algorithms to the framework.

\section{Evaluation Methodology}
\label{sec:interp_eval_methodology}
This section is dedicated to formulating a rigorous methodology of evaluatation. We start with the objective, which gives a brief introduction of what to be evaluated. Then two key evaluation questions are proposed, with detailed evaluation steps. Finally, the criteria and expected outcomes are presented so that it is rational to determine if the evaluation is indeed successful.

\subsection{Objective}
% [Develop a description (or access an existing version) of what is to be evaluated and how it is understood to work.]
This evaluation intends to validate that 1) the derived mapping from Chapter~\ref{ch:3DRecon_Mapping} can be extended to objects with different shapes, and demonstrate cases where it fails; 2) demonstrate the real-world use cases of the proposed framework. For the first goal, objects with varied degrees of shape changes are used, and the corresponding results are compared to the mapping. We attempt to demonstrate if the mapping, to some extent, is invariant to the changes of shape, and when would it fails to hold. For the second goal, we used real-world objects, and demonstrate if the framework can return a satisfactory result when provided with a correct description.

% \subsection{Frame}
% Set the parameters of the evaluation its purposes, key evaluation questions and the criteria and standards to be used.

% \subsubsection{Purpose}
% [What are the primary purposes and intended uses of the evaluation?]
% The evaluation intends to find out that the derived mapping can indeed find the algorithm that produces the best possible result from a suite of algorithms.

\subsection{Key Evaluation Questions and Steps}
% [What are the high level questions the evaluation will seek to answer? How can these be developed?]

The evaluation attempts to 1) \textit{prove that the derived mapping can be extended to other objects with different geometries}; 2). \textit{demonstrate that the framework can return a satisfactory reconstruction result given a correct description}.

\subsubsection{1. Extensiveness: does the mapping work for objects with a different shape?}
We first need to prove that the mapping derived in Chapter~\ref{ch:3DRecon_Mapping} is applicable to objects with different shapes. However, the variations of geometry is too vast and complicated to model, it wouldn't be possible to consider all these conditions. Thus we focus on one geometric property that in theory could have an impact on the mapping, which is the concavity of the surface. We use three synthetic objects with varied degrees of concavity, and see if the mapping is appliable under those circumstances, and when it would fail to hold. We use synthetic data to verify the mapping since it would not be practical to change material properties using real world objects.

The evaluation steps include:
\begin{itemize}
\item System setup: the synthetic data is generated by the Blender using the same setups in Chapter~\ref{ch:3DRecon_Mapping};
\item Data generation: we consider the six combinations of visual properties previously discussed in Chapter~\ref{ch:3DRecon_Taxo} since they encompass the majority of everyday objects;
\item Algorithm execution and evaluation: three selected algorithms as well as the baseline are used to reconstruct the synthetic object. Quantitative and qualitative results are plotted;
\item Validation of mapping: for each object, we verify if the reconstruction results are consistent to the mapping. If not, which algorithm is more suspectible to the change of concavity.
\end{itemize}

The outcome of a successful evaluation should be 1). the quantitative and qualitative results should be consistent to one another; 2). the techniques that work better than the baseline under each problem condition should be consistent to that of the mapping.

\subsubsection{2. Usefulness: can the framework return a satisfactory recontruction given the correct description.}
Given a correct description of the object, the algorithm chosen by the mapping should give the satisfactory reconstruction result. We use real-world dataset to test if this is indeed the case. However, the quantitative results are not available since we don't have the groundtruth models. Therefore, visual inspection is utilized to determine the quality of reconstruction model. The framework would choose the algorithm determined by the mapping, which is then compared to the baseline algorithm to determine if the quality is acceptable. As mentioned before, the baseline method is chosen so that it can always provide a decent reconstruction under most circumstances.

The evaluation steps are similar to those presented above except:
\begin{itemize}
\item System setup: the real-world data are captured using similar setups to the synthetic counterparts: for MVS, a Nikon D700 camera with [focal] lens are used; for photometric images, a Nikon D700 camera with [focal] lens, a handheld lamp, and two reference objects are used; for structured light techniques, a Nikon 700 camera and a [??] projector are used. We used nine everyday objects with varying texture, reflectance properties, and shape.
\item Validation of framework: demonstrate if a satisfactory result can be returned by the framework.
\end{itemize}

The outcome of a successful evalution should be 1). the framework should return the satisfactory result given a correct description, and a worse model given incorrect description; 2). If no one algorithm was selected, the framework should return the baseline result.

% \subsubsection{2. Usefulness of Mapping: How the mapping will return the result based on your description and your requirements}
% The purpose of the framework is not to compare which algorithm gives the best result, but to get the best possible reconstruction result given the correct description. Therefore, we want to see how well it works given the correct description, and how badly the result deteriorates given the incorrect description.

% The evaluation steps are:
% \begin{itemize}
% \item We chose three objects so that each algorithm would be activated by the mapping once as a demonstrative result.
% \item 
% \end{itemize}

% \subsubsection{2. Robustness: Does the mapping still return the best algorithm given an incorrect description?}
% Assume given a incorrect description of the object, will the mapping return a less satisfactory result instead, which is what it should behave like?

% \subsubsection{3. Improvement: Is the mapping more useful than the traditional approach?}
% Aside from being able to get the correct results, it's also important that the proposed approach has significant advantages over the traditional ones. We first need to identify the traditional way of employing reconstruction algorithms, find out the strengths and weakness and see if the proposed approach is superior than the exising one in the claimed aspects. The following are the aspects we set out to compare:
% \begin{itemize}
% \item is it easier?
% \item does it cater to more general object?
% \end{itemize}

% The evaluation steps are:
% \begin{itemize}
% \item Define the fundamental steps for both the traditional approach and the one proposed in the thesis.
% \item we adopt the same approach for analysing algorithmic complexity, and use basic step as the unit step to evaluate the complexity of using these two approaches. Complexity analysis is also a tool that allows us to explain how an algorithm behaves as the input grows larger. fundamental instructions/steps
% \end{itemize}

% \subsubsection{Criteria}
% Determine what `success' look like? What should be the criteria and standards for judging performance? Whose criteria and standards matter? What process should be used to develop agreement about these?

% \subsubsection{Steps}
% [Collect and retrieve data to answer descriptive questions about the activities of the project/programme/policy, the various results it has had, and the context in which it has been implemented.]

\section{Parameter Setting}
We provide results from three different descriptios where wach activates a different algorithm and provides a demonstrative result.
To address if the derived mapping works, The first step of the process is to estimate the amount of property in the object. We use a try-and-fit approach, where the user change the value of each property and see if the rendered result looks alike the real object. A similar approach can be found in the~\cite{Berkiten:2016:ARB} where the author also used a synthetic dataset to find the contributing factors of PS.
\begin{figure}[!htbp]
\centering
\begin{tabular}{cc}
  \includegraphics[width=0.5\textwidth]{interp/ui_sphere.PNG}&
  \includegraphics[width=0.5\textwidth]{interp/ui_teapot.PNG}\\
  (a) Lit sphere & (b) Lit teapot\\
\end{tabular}
\caption{The UI of determining the albedo, specular, and roughness of the surface. The albedo is set as around 0.8, which is determined by the value channel of HSV colour. The specular and roughness is set as 0.5, 0.2, respectively. (a) demonstrates the effect of the property setting on a sphere while (b) on a teapot.}
\label{fig:ui}
\end{figure}

\section{Extensiveness of Mapping}
\label{sec:interp_extend}
We first evaluate that the derived mapping does what it meant to do: return the best algorithm given a correct description of the object. The idea is that given the description of an arbitrary object, we use all three techniques for reconstruction, and see if the algorithm that has the best quantitative or qualitative result is consistent to the algorithm chosen by the mapping.

\subsection{Synthetic Datasets}
We use one object shown in Figure~\ref{fig:synth_data}, and four property settings in Table~\ref{tab:prop_list_synth_data} to test the validity of the abstraction. Those four settings represents four classes of objects discussed in Chapter~\ref{ch:3DRecon_Desc}. The best suited algorithm as suggested by the mapping derived from Chapter~\ref{ch:3DRecon_Mapping} is included.
\begin{table}[!htbp]
  \centering
  \begin{tabular}{l*{4}{c}|*{3}{c}}
  \hline
  & & & & & \multicolumn{3}{c}{Metrics}\\
  \textbf{Property} & Texture & Albedo & Specular & Roughness & Accuracy & Completeness & Ang diff\\
  \hline
  (a) & 0.2 & 0.8 & 0.2 & 0.8 & GSL & GSL & EPS\\
  (b) & 0.2 & 0.8 & 0.5 & 0.2 & GSL & - & - \\
  (c) & 0.8 & 0.8 & 0.2 & 0.8 & PMVS, GSL & PMVS, GSL & EPS \\
  (d) & 0.8 & 0.8 & 0.5 & 0.2 & PMVS, GSL & PMVS & -\\
  \hline
  \end{tabular}
  \caption{Property lists of the test objects.}
  \label{tab:prop_list_synth_data}
\end{table}

\begin{figure}[!htbp]
\centering
\begin{tabular}{cccc}
  \includegraphics[width=0.2\textwidth]{interp/synth_data/bottle/bottle_mvs}&
  \includegraphics[width=0.2\textwidth]{interp/synth_data/bottle/bottle_ps}&
  \includegraphics[width=0.2\textwidth]{interp/synth_data/bottle/bottle_sl}&
  \includegraphics[width=0.2\textwidth]{interp/synth_data/bottle/bottle_ps_gt}\\
  \includegraphics[width=0.2\textwidth]{interp/synth_data/knight/knight_mvs}&
  \includegraphics[width=0.2\textwidth]{interp/synth_data/knight/knight_ps}&
  \includegraphics[width=0.2\textwidth]{interp/synth_data/knight/knight_sl}&
  \includegraphics[width=0.2\textwidth]{interp/synth_data/knight/knight_ps_gt}\\
  \includegraphics[width=0.2\textwidth]{interp/synth_data/king/king_mvs}&
  \includegraphics[width=0.2\textwidth]{interp/synth_data/king/king_ps}&
  \includegraphics[width=0.2\textwidth]{interp/synth_data/king/king_sl}&
  \includegraphics[width=0.2\textwidth]{interp/synth_data/king/king_ps_gt}\\
  MVS & PS & SL & Normal groundtruth\\
\end{tabular}
\caption{The synthetic datasets and groundtruth for the first evaluation question. The three selected objects have different degrees of concavity. More specifically, the objects have increasing concavity.}
\label{fig:synth_data}
\end{figure}

% Here is the Table to test the effectiveness of the mapping
% \begin{figure}[!htbp]
% \centering
% \begin{tabular}{cccccc}
% \hline
% & & Mapping & \multicolumn{2}{c}{Accu\&Cmplt} & Norm\\
% \hline
% \multirow{2}{*}{Obj} & \multirow{2}{*}{Desc} & \multirow{2}{*}{Algo} & PMVS & Gray SL & Example PS\\
% \includegraphics[width=0.15\textwidth]{interp/synth_data/knight/knight_mvs} &
% 02080208 & EPS, GSL & 
% \includegraphics[width=0.15\textwidth]{interp/synth_data/knight/knight_mvs_02080208} &
% \includegraphics[width=0.15\textwidth]{interp/synth_data/knight/knight_ps_02080208} &
% \includegraphics[width=0.15\textwidth]{interp/synth_data/knight/knight_sl_02080208}\\
% \hline
% \end{tabular}
% \end{figure}

Now we show both the quantitative results and qualitative results of the test objects, and see if the results is consistent with the techniques selected by our abstraction. The result is shown in Figure~\ref{fig:synth_data_results}.
\begin{sidewaysfigure}[!htbp]
\centering
\begin{tabular}{c|ccccc}
  Mapping & Quantitative results & ~ & Qualitative results & ~\\
  \hline
  EPS, GSL & 
  \includegraphics[width=0.2\textwidth]{interp/synth_data/bottle/bottle_02080208}&
  \includegraphics[width=0.2\textwidth]{interp/synth_data/bottle/bottle_mvs_02080208.png}&
  \fcolorbox{green}{white}{\includegraphics[width=0.2\textwidth]{interp/synth_data/bottle/bottle_ps_02080208.png}}&
  \fcolorbox{green}{white}{\includegraphics[width=0.2\textwidth]{interp/synth_data/bottle/bottle_sl_02080208.png}}\\
  & \multicolumn{4}{c}{(a). tex(0.2), alb(0.8), spec(0.2), rough(0.8)}\\
  -&
  \includegraphics[width=0.2\textwidth]{interp/synth_data/bottle/bottle_02080502}&
  \includegraphics[width=0.2\textwidth]{interp/synth_data/bottle/bottle_mvs_02080502.png}&
  \fcolorbox{green}{white}{\includegraphics[width=0.2\textwidth]{interp/synth_data/bottle/bottle_ps_02080502.png}}&
  \includegraphics[width=0.2\textwidth]{interp/synth_data/bottle/bottle_sl_02080502.png}\\
  & \multicolumn{4}{c}{(b). tex(0.2), alb(0.8), spec(0.5), rough(0.2)}\\
  PMVS, EPS, GSL&
  \includegraphics[width=0.2\textwidth]{interp/synth_data/bottle/bottle_08080208}&
  \fcolorbox{green}{white}{\includegraphics[width=0.2\textwidth]{interp/synth_data/bottle/bottle_mvs_08080208.png}}&
  \fcolorbox{green}{white}{\includegraphics[width=0.2\textwidth]{interp/synth_data/bottle/bottle_ps_08080208.png}}&
  \fcolorbox{green}{white}{\includegraphics[width=0.2\textwidth]{interp/synth_data/bottle/bottle_sl_08080208.png}}\\
  & \multicolumn{4}{c}{(c). tex(0.8), alb(0.8), spec(0.2), rough(0.8)}\\
  PMVS&
  \includegraphics[width=0.2\textwidth]{interp/synth_data/bottle/bottle_08080502}&
  \fcolorbox{green}{white}{\includegraphics[width=0.2\textwidth]{interp/synth_data/bottle/bottle_mvs_08080502.png}}&
  \fcolorbox{green}{white}{\includegraphics[width=0.2\textwidth]{interp/synth_data/bottle/bottle_ps_08080502.png}}&
  \includegraphics[width=0.2\textwidth]{interp/synth_data/bottle/bottle_sl_08080502.png}\\
  & \multicolumn{4}{c}{(d). tex(0.8), alb(0.8), spec(0.5), rough(0.2)}\\
  \hline
  ~ & ~ & MVS & PS & SL\\
\end{tabular}
\caption{The first column shows the best algorithm chosen by the mapping. The quantitative and qualitative performance of each technique on the synthetic dataset. The red dots are from the ground truth while the black ones the reconstruction.}
\label{fig:synth_data_results}
\end{sidewaysfigure}

In this case of low concavity, we can see that the algorithms returned from the mapping is consistent to both the quantitative and qualitative results.

\begin{sidewaysfigure}[!htbp]
\centering
\begin{tabular}{c|ccccc}
  Mapping & Quantitative results & ~ & Qualitative results & ~\\
  \hline
  EPS, GSL & 
  \includegraphics[width=0.2\textwidth]{interp/synth_data/knight/knight_02080208}&
  \includegraphics[width=0.2\textwidth]{interp/synth_data/knight/knight_mvs_02080208.png}&
  \fcolorbox{green}{white}{\includegraphics[width=0.2\textwidth]{interp/synth_data/knight/knight_ps_02080208.png}}&
  \fcolorbox{green}{white}{\includegraphics[width=0.2\textwidth]{interp/synth_data/knight/knight_sl_02080208.png}}\\
  & \multicolumn{4}{c}{(a). tex(0.2), alb(0.8), spec(0.2), rough(0.8)}\\
  - &
  \includegraphics[width=0.2\textwidth]{interp/synth_data/knight/knight_02080502}&
  \includegraphics[width=0.2\textwidth]{interp/synth_data/knight/knight_mvs_02080502.png}&
  \includegraphics[width=0.2\textwidth]{interp/synth_data/knight/knight_ps_02080502.png}&
  \includegraphics[width=0.2\textwidth]{interp/synth_data/knight/knight_sl_02080502.png}\\
  & \multicolumn{4}{c}{(b). tex(0.2), alb(0.8), spec(0.5), rough(0.2)}\\
  PMVS, EPS, GSL&
  \includegraphics[width=0.2\textwidth]{interp/synth_data/knight/knight_08080208}&
  \fcolorbox{green}{white}{\includegraphics[width=0.2\textwidth]{interp/synth_data/knight/knight_mvs_08080208.png}}&
  \fcolorbox{green}{white}{\includegraphics[width=0.2\textwidth]{interp/synth_data/knight/knight_ps_08080208.png}}&
  \fcolorbox{green}{white}{\includegraphics[width=0.2\textwidth]{interp/synth_data/knight/knight_sl_08080208.png}}\\
  & \multicolumn{4}{c}{(c). tex(0.8), alb(0.8), spec(0.2), rough(0.8)}\\
  PMVS&
  \includegraphics[width=0.2\textwidth]{interp/synth_data/knight/knight_08080502}&
  \fcolorbox{green}{white}{\includegraphics[width=0.2\textwidth]{interp/synth_data/knight/knight_mvs_08080502.png}}&
  \includegraphics[width=0.2\textwidth]{interp/synth_data/knight/knight_ps_08080502.png}&
  \includegraphics[width=0.2\textwidth]{interp/synth_data/knight/knight_sl_08080502.png}\\
  & \multicolumn{4}{c}{(d). tex(0.8), alb(0.8), spec(0.5), rough(0.2)}\\
  \hline
  ~ & ~ & MVS & PS & SL\\
\end{tabular}
\caption{The first column shows the best algorithm chosen by the mapping. The quantitative and qualitative performance of each technique on the synthetic dataset. The red dots are from the ground truth while the black ones the reconstruction.}
\label{fig:synth_data_results}
\end{sidewaysfigure}

In this case of medium concavity, we can see that the algorithms returned from the mapping is consistent to both the quantitative and qualitative results.

\begin{sidewaysfigure}[!htbp]
\centering
\begin{tabular}{c|ccccc}
  Mapping & Quantitative results & ~ & Qualitative results & ~\\
  \hline
  GSL & 
  \includegraphics[width=0.2\textwidth]{interp/synth_data/king/king_02080208}&
  \includegraphics[width=0.2\textwidth]{interp/synth_data/king/king_mvs_02080208.png}&
  \includegraphics[width=0.2\textwidth]{interp/synth_data/king/king_ps_02080208.png}&
  \fcolorbox{green}{white}{\includegraphics[width=0.2\textwidth]{interp/synth_data/king/king_sl_02080208.png}}\\
  & \multicolumn{4}{c}{(a). tex(0.2), alb(0.8), spec(0.2), rough(0.8)}\\
  - &
  \includegraphics[width=0.2\textwidth]{interp/synth_data/king/king_02080502}&
  \includegraphics[width=0.2\textwidth]{interp/synth_data/king/king_mvs_02080502.png}&
  \includegraphics[width=0.2\textwidth]{interp/synth_data/king/king_ps_02080502.png}&
  \includegraphics[width=0.2\textwidth]{interp/synth_data/king/king_sl_02080502.png}\\
  & \multicolumn{4}{c}{(b). tex(0.2), alb(0.8), spec(0.5), rough(0.2)}\\
  PMVS, GSL&
  \includegraphics[width=0.2\textwidth]{interp/synth_data/king/king_08080208}&
  \fcolorbox{green}{white}{\includegraphics[width=0.2\textwidth]{interp/synth_data/king/king_mvs_08080208.png}}&
  \includegraphics[width=0.2\textwidth]{interp/synth_data/king/king_ps_08080208.png}&
  \fcolorbox{green}{white}{\includegraphics[width=0.2\textwidth]{interp/synth_data/king/king_sl_08080208.png}}\\
  & \multicolumn{4}{c}{(c). tex(0.8), alb(0.8), spec(0.2), rough(0.8)}\\
  PMVS&
  \includegraphics[width=0.2\textwidth]{interp/synth_data/king/king_08080502}&
  \fcolorbox{green}{white}{\includegraphics[width=0.2\textwidth]{interp/synth_data/king/king_mvs_08080502.png}}&
  \includegraphics[width=0.2\textwidth]{interp/synth_data/king/king_ps_08080502.png}&
  \includegraphics[width=0.2\textwidth]{interp/synth_data/king/king_sl_08080502.png}\\
  & \multicolumn{4}{c}{(d). tex(0.8), alb(0.8), spec(0.5), rough(0.2)}\\
  \hline
  ~ & ~ & MVS & PS & SL\\
\end{tabular}
\caption{The first column shows the best algorithm chosen by the mapping. The quantitative and qualitative performance of each technique on the synthetic dataset. The green dots are from the ground truth while the black ones the reconstruction.}
\label{fig:synth_data_results}
\end{sidewaysfigure}

In this case of high concavity, we can see that the algorithms returned from the mapping is no longer consistent to the quantitative and qualitative results for EPS.

\subsubsection{Results}
\textbf{(a), (b)} In Figure~\ref{fig:synth_data_results} (a), the mapping predicts that EPS and GSL can give satisfactory results, which is consistent to the quantitative result shown in column 2 and the qualitative resulted labeled in green rectangle. The completeness of the PMVS is low due to the lack of texture.

\textbf{(c), (d)} In Figure~\ref{fig:synth_data_results} (a), the mapping predicts that all three methods can give satisfactory results, which is consistent to the quantitative result shown in column 2 and the qualitative resulted labeled in green rectangle.

\section{Usefulness of Framework}
\label{sec:interp_useful}
Aside from testing whether the description would be correctly mapped to a satisfactory result, we should also verity if a less successful reconstruction would be return given an incorrect description.

\subsection{Real-world Datasets}
We used a similar setup to the synthetic settings and captured a real world dataset for nine objects. The property of these objects are listed in Table~\ref{tab:prop_list_real_data}. Since we don't have the ground truth, we resort to visual analysis to see if the algorithm gives the best reconstruction is consistent to the algorithm suggested by the mapping. We choose four representative objects as representatives of the six classes of objects, they are
\begin{figure}[!htbp]
\centering
\begin{tabular}{c|cccc}
\hline
class \# & 1 & 2 & 3\&4 & 5\&6\\
\hline
  & textureless & textureless & textured & textured\\
description & diffuse & mixed d/s & diffuse & mixed d/s\\
  & bright & bright & dark/bright & dark/bright\\
\hline
object & 
\raisebox{-.5\height}{\includegraphics[width=0.22\textwidth]{interp/real_world_obj/statue/statue}} &
\raisebox{-.5\height}{\includegraphics[width=0.22\textwidth]{interp/real_world_obj/cup/cup}} &
\raisebox{-.5\height}{\includegraphics[width=0.22\textwidth]{interp/real_world_obj/pot/pot}} &
\raisebox{-.5\height}{\includegraphics[width=0.22\textwidth]{interp/real_world_obj/vase/vase}}\\
\end{tabular}
\caption{The rerepsentatives of the six classes of objects used for evaluation.}
\label{fig:test_real_world_6class}
\end{figure}

We use the aforementioned methods to retrieve the parameters of each property, the decomposition of material for each object is presented in Figure~\ref{fig:real_data_material}.
% \begin{table}[!hbtp]
%   \centering
%   \begin{tabular}{*{12}{c}}
%   \multicolumn{3}{l}{\includegraphics[width=0.25\textwidth]{interp/real_world_obj/statue/statue}} &
%   \multicolumn{3}{l}{\includegraphics[width=0.25\textwidth]{interp/real_world_obj/cup/cup}} &
%   \multicolumn{3}{l}{\includegraphics[width=0.25\textwidth]{interp/real_world_obj/pot/pot}} &
%   \multicolumn{3}{l}{\includegraphics[width=0.25\textwidth]{interp/real_world_obj/vase/vase}}\\
%   \includegraphics[width=0.1\textwidth]{interp/real_world_obj/statue/base_00} & & &
%   \includegraphics[width=0.1\textwidth]{interp/real_world_obj/cup/base_00} & &
%   \includegraphics[width=0.1\textwidth]{interp/real_world_obj/pot/base_00} &
%   \includegraphics[width=0.1\textwidth]{interp/real_world_obj/pot/base_01} & &
%   \includegraphics[width=0.1\textwidth]{interp/real_world_obj/vase/base_00} &
%   \includegraphics[width=0.1\textwidth]{interp/real_world_obj/vase/base_01}\\
%   \multicolumn{3}{c}{(d). statue} & \multicolumn{3}{c}{(e). cup} & 
%   \multicolumn{3}{c}{(g). pot} & \multicolumn{3}{c}{(i). vase} \\
%   \end{tabular}
%   \caption{Material of Real-world objects.}
%   \label{fig:real_data_material}
% \end{table}

From the the decomposition of the material, we can have the property matrix shown in Table~\ref{tab:real_data_prop_list}.
\begin{table}[!htbp]
  \centering
  \begin{tabular}{l*{5}{c}}
  \hline
  \textbf{Property} & Texture & Albedo & Specular & Roughness & Best-suited Algo.\\
  \hline
  status & 0.2 & 0.8 & 0.2 & 0.5 & EPS, GSL\\
  cup & 0.2 & 0.8 & 0.2 & 0.2 & EPS, GSL\\
  pot & 0.8 & 0.2, 0.5 & 0.2 & 0.2 & PMVS\\
  vase & 0.8 & 0.8, 0.2 & 0.5 & 0.2 & PMVS\\
  \hline
  \end{tabular}
  \caption{Property list for the real-world objects}
  \label{tab:real_data_prop_list}
\end{table}

% In Figure~\ref{fig:test_real_world_obj}, we show the reconstructions of the real-world dataset. Since we don't have the ground truth, no quantitative measures are available, only visual inspection is used. We can show that the qualitative results match the results returned by the mapping.

\begin{figure}[!htbp]
\centering
\begin{tabular}{c|ccc|c}
& \multicolumn{3}{c}{Qualitative results} & Baseline\\
Mapping & PMVS & EPS & GSL & VH\\
\hline
EPS, GSL & 
\raisebox{-.5\height}{\includegraphics[width=0.2\textwidth]{interp/real_data/statue/statue_mvs}}&
\fcolorbox{green}{white}{\raisebox{-.5\height}{\includegraphics[width=0.2\textwidth]{interp/real_data/statue/statue_ps}}}&
\fcolorbox{green}{white}{\raisebox{-.5\height}{\includegraphics[width=0.2\textwidth]{interp/real_data/statue/statue_sl}}}&
\raisebox{-.5\height}{\includegraphics[width=0.2\textwidth]{interp/real_data/statue/statue_sc}}\\
EPS, GSL &
\raisebox{-.5\height}{\includegraphics[width=0.2\textwidth]{interp/real_data/cup/cup_mvs}}&
\fcolorbox{green}{white}{\raisebox{-.5\height}{\includegraphics[width=0.2\textwidth]{interp/real_data/cup/cup_ps}}}&
\fcolorbox{green}{white}{\raisebox{-.5\height}{\includegraphics[width=0.2\textwidth]{interp/real_data/cup/cup_sl}}}&
\raisebox{-.5\height}{\includegraphics[width=0.2\textwidth]{interp/real_data/cup/cup_sc}}\\
PMVS &
\fcolorbox{green}{white}{\raisebox{-.5\height}{\includegraphics[width=0.2\textwidth]{interp/real_data/pot/pot_mvs}}}&
\raisebox{-.5\height}{\includegraphics[width=0.2\textwidth]{interp/real_data/pot/pot_ps}}&
\raisebox{-.5\height}{\includegraphics[width=0.2\textwidth]{interp/real_data/pot/pot_sl}}&
\raisebox{-.5\height}{\includegraphics[width=0.2\textwidth]{interp/real_data/pot/pot_sc}}\\
PMVS &
\fcolorbox{green}{white}{\raisebox{-.5\height}{\includegraphics[width=0.2\textwidth]{interp/real_data/vase/vase_mvs}}}&
\raisebox{-.5\height}{\includegraphics[width=0.2\textwidth]{interp/real_data/vase/vase_ps}}&
\raisebox{-.5\height}{\includegraphics[width=0.2\textwidth]{interp/real_data/vase/vase_sl}}&
\raisebox{-.5\height}{\includegraphics[width=0.2\textwidth]{interp/real_data/vase/vase_sc}}\\
\hline
\end{tabular}
\caption{The evaluation of the effectiveness of the mapping using real-world object. The well reconstructed object is label by green rectangle.}
\end{figure}

\section{Summary}

